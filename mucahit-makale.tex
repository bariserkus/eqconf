%% LyX 2.3.2-2 created this file.  For more info, see http://www.lyx.org/.
%% Do not edit unless you really know what you are doing.
\documentclass[12pt,turkish]{article}
\usepackage{mathptmx}
\renewcommand{\familydefault}{\rmdefault}
\usepackage[T1]{fontenc}
\usepackage[utf8]{inputenc}
\usepackage[a4paper]{geometry}
\geometry{verbose,tmargin=25mm,bmargin=20mm,lmargin=25mm,rmargin=30mm,footskip=10mm}
\usepackage{fancyhdr}
\pagestyle{fancy}
\usepackage{amsmath}
\usepackage{amssymb}
\usepackage{stackrel}
\usepackage{graphicx}
\usepackage{setspace}
\onehalfspacing

\makeatletter
%%%%%%%%%%%%%%%%%%%%%%%%%%%%%% User specified LaTeX commands.
%\usepackage{titling}
%\usepackage{newtxtext,newtxmath}
%\usepackage{mathptmx}
%\usepackage{adjustbox}
%\usepackage{caption}
%\usepackage{showframe}
%\usepackage{strtikz}
%\usepackage{ifthen}
\usepackage{titling}

%%%% Tikz Packages %%%%
%\pgfplotsset{compat=1.14}
%\usepackage{tikz}
%\usetikzlibrary{shapes,arrows.meta, graphs, graphs.standard}
%\usetikzlibrary{math, fit}
%\usetikzlibrary{calc,intersections,through,backgrounds,decorations.pathmorphing}
%\usetikzlibrary{external}

%%%% BibLaTeX
%If BibLaTeX is used, some of the bib terms can be converted to Turkish
\usepackage{etoolbox}
\AtEndPreamble{%
	\DefineBibliographyStrings{english}{%
 		references = {{Kaynaklar}},
		and = {ve},
        andothers = {vd.},
	}
    \renewbibmacro{in:}{}
	\usepackage{csquotes}
}

% To fix the bug that appears when scaling graphics.
% See https://tex.stackexchange.com/questions/32178/usepackageturkishbabel-and-includegraphics-inconcistency
%\AtBeginDocument{
%\shorthandoff{=}
%}

%%% Some commands for setting the spcaings
%\renewcommand{\arraystretch}{2}
%\setlength\parindent{0pt}
%\captionsetup[table]{skip=10pt}
%\setlength{\tabcolsep}{2pt}

%%% Space for equations
%\makeatletter
%\g@addto@macro \normalsize {%
% \setlength\abovedisplayskip{8pt}%
% \setlength\belowdisplayskip{8pt}%
%}
%\makeatother

%\date{}

%%%% Options for titling package. Makes the title bold
\pretitle{\begin{center} \Large \bf }
\posttitle{\par\end{center}}
\preauthor{\begin{center} \large \begin{tabular}[t]{c}}
\postauthor{\end{tabular}\par\end{center}}

\rhead{Bekin ve Erkuş}
\lhead{Ortak Yalıtımlı Hastaneler}

%%% To have different header for the first page.
\fancypagestyle{firststyle}
{
\lhead{1. Taslak}
\chead{}
\rhead{Gönderilen Dergi: }
}

%%% Matrix Greek Letters
\usepackage{upgreek}   %Needed for Matrix Greek Letters
\usepackage{pdfrender} %Needed for Matrix Greek Letters
\newcommand*{\boldgreek}[1]{%
  \textpdfrender{%
    TextRenderingMode=FillStroke,%
    LineWidth=.35pt,%
  }{#1}%
}

\makeatother

\usepackage{babel}
\usepackage{xkeyval}

\usepackage[style=authoryear,maxcitenames=2, doi=false,isbn=false,url=false, eprint=false, giveninits]{biblatex}
\addbibresource{baseiso.bib}
\addbibresource{codes.bib}
\addbibresource{stranalysis.bib}
\begin{document}
\title{Ortak Yalıtım Düzleminde Bulunan Yalıtımlı Hastane Yapılarının İncelenmesi}
\author{Mücahit Bekin\thanks{Doktora Öğrencisi, İnşaat Müh. Böl. İstanbul Teknik Üniv.; bekinm@itu.edu.tr}
\and Barış Erkuş\thanks{Dr. Öğr. Üye., İnşaat Müh. Böl. İstanbul Teknik Üniv.; bariserkus@itu.edu.tr}}
\date{10 Şubat 2019}

\maketitle
\thispagestyle{firststyle}
\shorthandoff{=}
\begin{abstract}
Özet.
\end{abstract}


\section{Giriş}

Sismik yalıtımlı bina yapılarında yalıtıcılar, üstyapı olarak adlandırılan
bina yapısı ile temel arasında rijitliği yapıya göre çok daha düşük
olan bir katman oluşturur. Bu katman sayesinde yalıtımlı yapılarda
etkin salınım mod şekli, yanal yerdeğiştirmelerin yalıtım seviyesinde
üstyapıya göre yüksek olduğu ve üstyapının toplu kütle gibi davrandığı
bir hal alır. Bundan dolayı, yalıtımlı bir yapının etkin doğal salınım
mod periyodu, aynı yapının yalıtımsız halinin periyoduna göre daha
yüksektir. Periyot uzaması olarak da adlandırılan bu durum, yalıtım
seviyesinden yapıya aktarılan taban kesme kuvvetlerinin yine aynı
yapının yalıtımsız haline göre daha küçük olmasını sağlar. Buna ek
olarak, yalıtım seviyesinde oluşan yerdeğiştirmelerin kontrolü ve
deprem sırasında oluşacak iç kuvvetlerin azalım göstermesi için yalıtım
katmanının enerji sönümleme kabiliyeti olması gerekir. Kurşun çekirdekli
elastomer yalıtıcılar ve sürtünmeli sarkaç yalıtıcılar, enerji sönümlemesini
yapı mukavemetine göre daha düşük olan akma seviyelerine ve yüksek
yerdeğiştirmelere sahip histeretik davranış üzerinden kendi bünyelerinde
barındırır. Yüksek sönümlemeli elastomer mesnetler viskoelastik davranışa
sahiptirler. Diğer bir yaklaşım, enerji sönümleme kabiliyeti düşük
olan ve doğrusala yakın davranış gösteren elastomer mesnetler ile
sönümleyicilerin beraber kullanılmasıdır. İlk yaklaşım, uygulamalarda
kolaylıklar sağlarken, ikinci yaklaşım daha karmaşık ve ileri tasarım
hedeflerine ulaşılmasında tercih edilir. Uzun yıllar yapılan araştırmalar
sayesinde farklı tip yalıtım birimlerinin ve yalıtımlı yapıların davranışları
oldukça iyi anlaşılmıştır (\cite{Skinner1993}). Günümüzde, bu yapıların
tasarımları ile ilgili birçok yönetmelik mevcut olup (örnek: \cite{ASCE7-16,TBDY2018})
hem dünyada, hem de özellikle son yıllarda ülkemizde, birçok uygulama
mevcuttur . 

\end{document}
