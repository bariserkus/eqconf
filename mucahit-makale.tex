%% LyX 2.3.2-2 created this file.  For more info, see http://www.lyx.org/.
%% Do not edit unless you really know what you are doing.
\documentclass[12pt,turkish]{article}
\usepackage{mathptmx}
\renewcommand{\familydefault}{\rmdefault}
\usepackage[T1]{fontenc}
\usepackage[utf8]{inputenc}
\usepackage[a4paper]{geometry}
\geometry{verbose,tmargin=25mm,bmargin=20mm,lmargin=25mm,rmargin=30mm,footskip=10mm}
\usepackage{fancyhdr}
\pagestyle{fancy}
\usepackage{amsmath}
\usepackage{amssymb}
\usepackage{stackrel}
\usepackage{graphicx}
\usepackage{setspace}
\onehalfspacing

\makeatletter
%%%%%%%%%%%%%%%%%%%%%%%%%%%%%% User specified LaTeX commands.
%\usepackage{titling}
%\usepackage{newtxtext,newtxmath}
%\usepackage{mathptmx}
%\usepackage{adjustbox}
%\usepackage{caption}
%\usepackage{showframe}
%\usepackage{strtikz}
%\usepackage{ifthen}
\usepackage{titling}

%%%% Tikz Packages %%%%
%\pgfplotsset{compat=1.14}
%\usepackage{tikz}
%\usetikzlibrary{shapes,arrows.meta, graphs, graphs.standard}
%\usetikzlibrary{math, fit}
%\usetikzlibrary{calc,intersections,through,backgrounds,decorations.pathmorphing}
%\usetikzlibrary{external}

%%%% BibLaTeX
%If BibLaTeX is used, some of the bib terms can be converted to Turkish
\usepackage{etoolbox}
\AtEndPreamble{%
	\DefineBibliographyStrings{english}{%
 		references = {{Kaynaklar}},
		and = {ve},
        andothers = {vd.},
	}
    \renewbibmacro{in:}{}
	\usepackage{csquotes}
}

% To fix the bug that appears when scaling graphics.
% See https://tex.stackexchange.com/questions/32178/usepackageturkishbabel-and-includegraphics-inconcistency
%\AtBeginDocument{
%\shorthandoff{=}
%}

%%% Some commands for setting the spcaings
%\renewcommand{\arraystretch}{2}
%\setlength\parindent{0pt}
%\captionsetup[table]{skip=10pt}
%\setlength{\tabcolsep}{2pt}

%%% Space for equations
%\makeatletter
%\g@addto@macro \normalsize {%
% \setlength\abovedisplayskip{8pt}%
% \setlength\belowdisplayskip{8pt}%
%}
%\makeatother

%\date{}

%%%% Options for titling package. Makes the title bold
\pretitle{\begin{center} \Large \bf }
\posttitle{\par\end{center}}
\preauthor{\begin{center} \large \begin{tabular}[t]{c}}
\postauthor{\end{tabular}\par\end{center}}

\rhead{Bekin ve Erkuş}
\lhead{Ortak Yalıtımlı Hastaneler}

%%% To have different header for the first page.
\fancypagestyle{firststyle}
{
\lhead{1. Taslak}
\chead{}
\rhead{Gönderilen Dergi: }
}

%%% Matrix Greek Letters
\usepackage{upgreek}   %Needed for Matrix Greek Letters
\usepackage{pdfrender} %Needed for Matrix Greek Letters
\newcommand*{\boldgreek}[1]{%
  \textpdfrender{%
    TextRenderingMode=FillStroke,%
    LineWidth=.35pt,%
  }{#1}%
}

\makeatother

\usepackage{babel}
\usepackage{xkeyval}

\usepackage[style=authoryear,maxcitenames=2, doi=false,isbn=false,url=false, eprint=false, giveninits]{biblatex}
\addbibresource{baseiso.bib}
\addbibresource{codes.bib}
\addbibresource{stranalysis.bib}
\begin{document}
\title{Ortak Yalıtım Düzleminde Bulunan Yalıtımlı Hastane Yapılarının İncelenmesi}
\author{Mücahit Bekin\thanks{Doktora Öğrencisi, İnşaat Müh. Böl. İstanbul Teknik Üniv.; bekinm@itu.edu.tr}
\and Barış Erkuş\thanks{Dr. Öğr. Üye., İnşaat Müh. Böl. İstanbul Teknik Üniv.; bariserkus@itu.edu.tr}}
\date{10 Şubat 2019}

\maketitle
\thispagestyle{firststyle}
\shorthandoff{=}
\begin{abstract}
Özet.
\end{abstract}


\section{Giriş}

Sismik yalıtımlı bina yapılarında yalıtıcılar, üstyapı olarak adlandırılan
bina yapısı ile temel arasında rijitliği yapıya göre çok daha düşük
olan bir katman oluşturur. Bu katman sayesinde yalıtımlı yapılarda
etkin salınım mod şekli, yanal yerdeğiştirmelerin yalıtım seviyesinde
üstyapıya göre yüksek olduğu ve üstyapının toplu kütle gibi davrandığı
bir hal alır. Bundan dolayı, yalıtımlı bir yapının etkin doğal salınım
mod periyodu, aynı yapının yalıtımsız halinin periyoduna göre daha
yüksektir. Periyot uzaması olarak da adlandırılan bu durum, yalıtım
seviyesinden yapıya aktarılan taban kesme kuvvetlerinin yine aynı
yapının yalıtımsız haline göre daha küçük olmasını sağlar. Buna ek
olarak, yalıtım seviyesinde oluşan yerdeğiştirmelerin kontrolü ve
deprem sırasında oluşacak iç kuvvetlerin azalım göstermesi için yalıtım
katmanının enerji sönümleme kabiliyeti olması gerekir. Kurşun çekirdekli
elastomer yalıtıcılar ve sürtünmeli sarkaç yalıtıcılar, enerji sönümlemesini
yapı mukavemetine göre daha düşük olan akma seviyelerine ve yüksek
yerdeğiştirmelere sahip histeretik davranış üzerinden kendi bünyelerinde
barındırır. Yüksek sönümlemeli elastomer mesnetler viskoelastik davranışa
sahiptirler. Diğer bir yaklaşım, enerji sönümleme kabiliyeti düşük
olan ve doğrusala yakın davranış gösteren elastomer mesnetler ile
sönümleyicilerin beraber kullanılmasıdır. İlk yaklaşım, uygulamalarda
kolaylıklar sağlarken, ikinci yaklaşım daha karmaşık ve ileri tasarım
hedeflerine ulaşılmasında tercih edilir. Uzun yıllar yapılan araştırmalar
sayesinde farklı tip yalıtım birimlerinin ve yalıtımlı yapıların davranışları
oldukça iyi anlaşılmıştır (\cite{Skinner1993}). Günümüzde, bu yapıların
tasarımları ile ilgili birçok yönetmelik mevcut olup (örnek: \cite{ASCE7-16,TBDY2018})
hem dünyada, hem de özellikle son yıllarda ülkemizde, birçok uygulama
mevcuttur . 

Yalıtımlı bina yapılarının tasarımında genelde üç aşamalı bir yol
izlenir. İlk aşama kendi içerisinde iki bölümden oluşur. İlk aşamada
ilk bölümün amacı öngörülen üstyapı boyutlarına bağlı olarak yalıtım
birimlerinin ön tasarımını gerçekleştirmektir. Bu bölümde tüm yapı,
üstyapının toplu bir kütle ve tüm yalıtıcıların ise bir yay olarak
kabul edildiği tek serbestlik dereceli bir sisteme (TSDS) indirgenir.
TSDS'deki yay, viskoelastik yalıtıcılar ise doğrusal bir yaydır ve
yalıtıcının sönümlemesi viskoz sönümleme ile ifade edilir. Bu yay,
histeretik yalıtıcılar için çiftdoğrusal davranışa sahip doğrusal
olmayan bir yaydır ancak aşağıda anlatıldığı üzere yöntem içerisinde
bu doğrusal olmayan yay da eşdeğer bir doğrusal yay ve eşdeğer bir
viskoz sönümlemenin üstdüşümüne indirgenir. Bu noktada amaç, yapı
lokasyonu için yönetmeliklerin öngördüğü ya da sahaya özel bir depremsellik
çalışmasından elde edilen spektrum kullanılarak TSDS'in tepkilerini
elde etmektir. Doğrusal bir yay ve viskoz sönümlemeye sahip bir TSDS'in
periyodu, yerdeğiştirmesi ve yay kuvveti, verilen spektrum için temel
dinamik bilgileri kullanılarak rahatça hesaplanabilir. Spektrumlar
genelde \% 5 kritik sönümleme için verildiğinden, ilgili spektrum,
yönetmeliklerde verilen sönüm değiştirme katsayıları kullanılarak,
TSDS'in viskoz sönümleme oranına uygun hale getirilir. Bu yöntem,
viskoelastik yalıtıcılarda doğrudan uygulanabilirken, histeretik yalıtıcılarda
çiftdoğrusal yay ile sönümlenen enerjinin eşdeğer bir viskoz sönümleme
ile ifadesini gerektirir. Bu amaçla, yapının öngörülen yerdeğiştirme
genliğinde sinüzoidal bir salınım yaptığı kabulü kullanılır. Eşdeğer
sönüm, eşdeğer viskoz elemanın öngörülen genlikte sönümlendiği enerjinin
çift doğrusal elemanın aynı genlikte yaptığı bir çevrim ile sönümlediği
enerjiye eşitlenmesi ile elde edilir. Bu yaklaşımdan dolayı, TSDS
tepkilerinin hesabı histeretik yalıtıcılar için yinelemeli bir hal
alır.

İlk aşamanın ikinci bölümünde amaç, ilk bölüme esas teşkil eden üstyapı
kütlesinin teyidi ve buna bağlı olarak üstyapı ön tasarımının yapılmasıdır.
Bu bölümde, ilk bölümden elde edilen yay kuvvetleri, üstyapıya aktarılan
taban kesme kuvvetleri olduğu kabulü ile, tipik eşdeğer statik yöntemlerde
olduğu gibi üstyapıya dağıtılır. Bu dağıtım yönetmeliklere bağlı olarak
farklı şekillerde yapılabilir. Dağıtımdan sonra statik analizler ile
üstyapı tipik yapılar gibi tasarlanır. İlk aşama, farklı yalıtıcı
tipleri ya da aynı yalıtıcı tipinin farklı üreticilerden gelen özellikleri
için tekrarlanabilir. 

İkinci bölümde amaç, dinamik yöntemler kullanarak üstyapının daha
kapsamlı tasarımını yapmak ve bu çalışma sırasında yalıtıcı tasarımının
teyididir. Bu amaçla genelde yalıtımlı yapının üç boyutlu modeli kullanılır.
Bu modelde yalıtıcılar ilk bölümde elde edilen yay özellikleri kullanılarak
modellenir. Yalıtıcıların viskoz sönümleme özellikleri ise, spektrumun,
TSDS'in salınım periyodunun üzerinde kalan periyotlar için spektrumun
sönüm değiştirme katsayısı kullanılarak güncellenmesi neticesinde
analizlere yansır. Mod birleştirme yöntemi kullanılarak yapılan analizler
ile üstyapı ve altyapı tasarımı yapılır. Bu noktada, her ne kadar
yönetmelikler deprem azaltma katsayılarının kullanılmasına izin verse
de, yalıtımlı yapı teorisi çoklukla üstyapının doğrusal davranış gösterdiği
kabulüne dayandığından, tasarım maksadı ile bu katsayıların kullanılması
tavsiye edilmez. Bu katsayılar daha çok, üçüncü aşamada karşılaşılacak
olan yapı stabilitesi için çok kritik olmayan elemanlarda oluşan düşük
seviyedeki doğrusal olmayan davranışlar için kullanılmaktadır.

Birinci ve ikinci aşamalarda üstyapı tasarımı genelde tasarım depremi
olarak adlandırılan ve 50 yılda aşılma olasılığı \%10 olan deprem
seviyesi için ilgili yük kombinasyonları ve malzeme faktörleri kullanarak
yapılır. Ancak, aşağıda açıklanacağı üzere, yalıtıcı yerdeğiştirmelerini
en büyük deprem olarak adlandırılan ve 50 yılda aşılma olasılığı \%2
olan deprem seviyesi için hesaplamak ve yalıtıcı seçimini bu yerdeğiştirmeler
üzerinden yapmak gerekmektedir. 

Üçüncü ve son bölümde amaç tasarımı tamamlanan yalıtımlı yapının performans
değerlendirmesini yapmaktır. Bu değerlendirme ile üstyapı hedef performans
seviyelerini sağlamalı ve yalıtıcıların tasarımı teyit edilmelidir.
Bu amaçla genelde kapsamlı doğrusal olmayan analizler kullanılır.
Bu analizlerde yalıtıcılar, gerçek davranışlarını en iyi yansıtan
doğrusal olmayan bünye modelleri kullanılarak modellenir. Bu aşamada
genelde en büyük deprem seviyesi kullanılmalıdır. Bunun nedeni, deprem
yönetmeliklerinin tüm yapılar için en temel şartı, en büyük deprem
altında ilgili yönetmeliğe göre tasarlanan yapının en az göçme öncesi
tabir edilen performans seviyesini göstermesidir. Tipik yapılar için
bu şartın, tasarım depremi ve yönetmeliklerde verilen sünek detayların
ve ilgili deprem azaltma katsayıların kullanılması durumunda sağlandığı
kabul edilir. Ancak, tipik yapılarda doğrusal olmayan davranış, yapısal
elemanlarda olurken, yalıtımlı yapılarda doğrusal olmayan davranış
yalıtıcılarda olmaktadır ve bu yüzden yalıtımlı yapılar tipik yapılardan
farklıdır. Buna ek olarak, önceden de bahsedildiği gibi yalıtımlı
yapı teorisi üstyapının doğrusal olması kabulüne dayanmaktadır ve
üstyapı tasarımında deprem azaltma katsayısı etkin olarak kullanılmışsa,
en yüksek depremde yapının ileri seviyede doğrusal olmayan davranış
göstermesi beklenebilir. Tüm bu nedenlerden dolayı performans hedeflerinin
değerlendirmesini en büyük deprem için tanımlamak ve bu deprem seviyesi
için performans değerlendirme analizlerini gerçekleştirmek gerekir.

Yalıtımlı yapı uygulamalarının büyük çoğunluğu tekil üstyapı, üstyapının
üzerine oturduğu ve alt yüzeyinden yalıtıcıların üst noktalarına bağlanan
yalıtım diyaframı ya da döşemesi, bu döşemenin altında bulunan yalıtım
katmanı ve bu katman altında bulunan yalıtıcıların alt noktalarının
bağlandığı altyapı sistemi şeklindedir (Şekil \ref{fig:typ_vs_common}a).
Bundan farklı olarak daha geniş bir yalıtım düzlemi ve yalıtım döşemesi
üzerinde birden fazla yapının bulunduğu yapılar da mevcuttur (Şekil
\ref{fig:typ_vs_common}b). Bu yapılar, ortak yalıtım düzleminde bulunan
yalıtımlı yapı olarak adlandırılmaktadır. Ortak yalıtım düzlemine
sahip yalıtımlı yapı uygulamalarının dünyadaki örnekleri olarak 
gösterilebilir. Ülkemizde özellikle son yıllarda sayıları artan ve
hastane kampüsleri olarak bilinen projelerde bu forma sahip yalıtımlı
yapılar olduğu bilinmektedir. Bu projeler büyük ölçekli projeler olup
bina sayısı onlar mertebesindedir. Her bir hastane bina yapının tipik
hastane bina yapısı olduğu düşünülürse, tüm yalıtımlı yapı boyutunun
büyüklüğü daha rahatça anlaşılabilir. Bu yapılarda yalıtım diyaframları
300 m ila 400 m boyutlarına ve yalıtıcı sayıları 2000 ila 3000 mertebelerine
ulaşabilmektedir. 
\begin{figure}
\noindent \begin{centering}
\includegraphics{figures/tikz/typ_baseiso}\bigskip{}
\par\end{centering}
\noindent \centering{}\includegraphics{figures/tikz/common_baseiso}\caption{\label{fig:typ_vs_common}Yalıtımlı yapılarda üstyapı yerleşimlerinin
şematik gösterimi.}
\end{figure}

Ortak yalıtım düzlemine sahip yalıtımlı yapılar için çeşitli çalışmalar
mevcuttur.  tarafından yapılan çalışmalarda, bu forma sahip yapılar
için bir analiz programı geliştirilmiş ve  

Ortak yalıtım düzlemine sahip yalıtımlı yapıların tipik tekil yalıtımlı
yapılara göre bazı avantaj ve dezavantajları bulunmaktadır. Bu yapı
formu özellikle birbirlerine yakın olan ve birbirleri arasında mimari
kullanım için geçişleri bulunan birçok yapının bir arada bulunduğu
projelerde yalıtım teknolojisi kullanılmak istenirse faydalı olmaktadır.
Bu yapıların tekil yalıtımlı yapı olmaları durumunda oluşacak göreli
yerdeğiştirmeler ve olası faz farkları, yapılar arasındaki mimari
geçişlerin tasarımını ve uygulamasını oldukça zorlaştırmaktadır. Ortak
yalıtım düzlemi ile mimari geçişler yalıtımsız binalarda olduğu gibi
yapılabilmektedir. Olumsuz yönler olarak, inşaat açısından bakıldığında,
yalıtım diyaframının daha yüksek sünme rötre etkilerine maruz kalacağı,
diyaframın inşaasının daha zor olacağı, çok sayıda yalıtıcının üretilmesi
ve üstyapıların inşaasının yalıtım katmanı ve yalıtım diyaframına
bağlı olması gibi konular gösterilebilir.

Ortak yalıtım düzlemine sahip yalıtımlı yapıların tasarımı ve analizi
de önemli zorluklar içermektedir. İlk olarak, bu yapıların tasarımına
ilişkin genel kabul görmüş bir yaklaşım ve/veya yönetmelik mevcut
değildir. Yalıtımlı yapı yönetmelikleri, yalıtımlı yapı teorisinin
ve bu konuda yapılan çalışmaların çoğunlukla tekil yapılar üzerine
olmasından dolayı, tekil yapılar için hazırlanmıştır. Özellikle yalıtıcı
ve üstyapı ön tasarımında kullanılan ve tekil yapıyı TSDS'e indirgeyen
yaklaşım ortak yalıtımlı yapılarda uygulanamaz. Çok kaba yaklaşımlar
ile ön tasarım yapılsa bile, gerçek davranışın bu tasarımdan oldukça
farklı olabileceği, çeşitli çalışmalar ile gösterilmiştir . Sonuç
olarak, bu yapılarda yapılması zorunlu olan, tüm yalıtımlı yapıların
bulunduğu sistemin kapsamlı bir modelde yapılarda oluşabilecek faz
farklarını da göz önüne alarak ileri analizlere tabi tutmaktır. Ancak
yapı sayısının fazla olması ve üstyapı modellerinin de kapsamlı olması
durumunda, bu tür analizlerin de başarı ile tamamlanması, analiz programında
veri alımı ve yapısal elemanların alınan veri ile tasarımının yapılması
da oldukça zor olmaktadır. Üstyapılar için toplu kütle modelleri kullanmak
gibi yaklaşımlar ile tüm yapı modeli daha basit bir modele indirgenebilir.
Bu tür bir yaklaşım, yalıtıcı ve kat kesme kuvvetlerinin hesabına
uygun olsa da, bu yaklaşım ile tasarım için gerekli olan eleman kuvvetlerinin
elde edilmesi, üstyapıda doğrusal olmayan davranış bekleniyor ise,
bu davranışların modellenmesi ya da yalıtım seviyesi düzlemi için
diyafram kuvvetlerinin gerçekçi bir şekilde elde edilmesi mümkün değildir.

Ülkemizdeki hastane projelerinde izlenen analiz ve tasarım yöntemi
olarak, TSDS kabulünün kullanıldığı anlaşılmaktadır. Yalıtım düzlemi
üstünde kalan yalıtım diyaframı ve hastane bina yapılarınının toplam
kütlesi hesaplanarak bunun TSDS'in kütlesi olduğu kabul edilmektedir.
Yukarıda açıklanan TSDS analizleri yapılarak yalıtım seviyesi kuvvetleri
bulunmakta ve bu kuvvetler daha sonra üstyapılara, yapıların kütleleri
oranında taban kesme kuvveti olarak dağıtılmaktadır. Üstyapılar bu
kuvvetler altında tasarlanmaktadır. Tüm üstyapıları içeren tümleşik
bir modelin oluşturulması ve bu tümleşik modelin mod birleştirme yöntemi
ile analizi yapılabilmektedir. Ancak, tümleşik yapıda doğrusal olmayan
zaman-tanım alanı analizleri yapılmamakta, yapı kat kesme kuvvetlerinde
oluşan dinamik artımlar gözönüne alınmamaktadır. Literatürde, ortak
yalıtım düzleminde bulunan yalıtımlı yapılar hakkında bazı nümerik
çalışmalar mevcut olmakla beraber, ülkemizde yapılmakta olan ve çok
büyük ölçekli ve çok fazla sayıda üstyapıya sahip bu tür yapıların
tümleşik halinin incelendiği çalışmalar bulunmamaktadır. Çok yapılı
yalıtımlı sistemlerin ülkemize özel hastane yapı uygulamalarının kapsamlı
incelenmesi literatüre önemli bir katkı olacaktır. Hastane projelerinde,
üstyapı sayısı ve her yapının kat adedi çok olduğundan, bu tür bir
çalışmanın sayısal bazlı olması uygun olacaktır.

Bu çalışmada ülkemizde uygulamaları bulunan ortak yalıtım düzlemine
sahip hastane yapılarının ve bu yapıların analizinde kullanılan yöntemlerin
sayısal analiz ile kapsamlı parametrik incelenmesi yapılmıştır. Bu
amaçla, ülkemizde inşaa edilmekte olan hastane projelerine benzer
bir tümleşik bir yapı kurgulanmıştır. Bu yapıda birçok farklı yapısal
periyotlara sahip üstyapı bulunmaktadır. Bu yapıların her biri ticari
bir analiz programında modellenerek, kütle, rijitlik ve salınım periyodu
bilgileri elde edilmiştir. Tümleşik yapı, literatürde sıkça kullanılan
ve üstyapının toplu kütleler ile, yalıtıcıların toplu bir çiftdoğrusal
elaman ile, yalıtım döşemesinin rijit bir kütle ile modellendiği bir
yaklaşımla modellenerek dinamik hareket denklemleri elde edilmiştir.
Bu hareket denklemleri yazılan bir \textsc{Matlab} programı ile öngörülen
bir grup deprem kaydı için çözülmüş, ve yapısal kuvvetler elde edilmiştir.
Bu kuvvetler, ülkemizde bu yapı tipleri için uygulanan yaklaşım yönteminden
elde edilen kuvvetler ile karşılaştırılarak, yapısal davranış irdelenmiştir.


\section{Teorik Altyapı}

Bu bölümde, ilk olarak tekil ve çoklu yalıtımlı yapıların doğrusal
olmayan zaman-tanım analizlerinde kullanılan hareket denklemleri ve
bu hareket denkleminin çözüm yöntemleri anlatılmıştır. Daha sonra
tipik tekil yalıtımlı yapıların TSDS yaklaşımı ile tasarımı ve bu
yaklaşımın ülkemizde çoklu yapıların tasarımında nasıl kullanıldığı
açıklanmıştır. Ayrıca bu makale kapsamında kullanılacak diğer yöntemler
hakkında da bilgiler verilmiştir. Denklemlerin detayları ilgili literatürde
detaylı verildiğinden, burada genel halleri verilmiştir.

\subsection{Sismik Yalıtımlı Yapı Modellenmesi ve Analizi}

Sismik yalıtımlı yapılar farklı yöntemler ile modellenebilir. Mühendislik
uygulamalarında yapının tüm bileşenlerinin tasarımı gerektiğinden,
genelde tüm yapının detaylı sonlu elemanlar yaklaşımı ile modellenmesi
tercih edilmektedir (örnek: \cite{Zekioglu2009}). Diğer bir yaklaşım,
üstyapı, yalıtım diyaframı ve yalıtım katmanı hareket denklemlerinin
ayrı ayrı elde edilmesi ve sonradan birleştirilmesi üzerine kuruludur
\parencite{Nagarajaiah1991}. Bu yöntemde üstyapı doğrusal bir modelle,
yalıtım diyaframı toplu kütle ile, yalıtım birimleri ise çoklu ya
da tekil doğrusal olmayan yaylar ile modellenebilir. Üstyapı modeli
üç boyutlu bir modelin modal uzayda ifadesi ya da toplu kütle ve kayma
yaylarından oluşan bir modelle ifade edilebilir. Üstyapı için kapsamlı
bir modal uzay modeli, üstyapının mevcut üç boyutlu modeli olması
durumunda üstyapı tasarımına yönelik olarak eleman kuvvetlerinin elde
edilmesi amaçlı kullanılabilir (örnek: \cite{Narasimhan2006,Erkus2006}).
Toplu kütle-yay modeli ise bir davranış biçiminin nümerik incelenmesi
ile ilgili çalışmalarda kullanılabilir (örnek: \cite{Bekin2018-MSThesis}).

Bilindiği üzere, yalıtımsız yapılarda toplu kütle-modeli kullanılması
durumunda kütle ve yay özellikleri seçilirken genelde, mevcut ya da
tipik bir yapının, kütle, rijitlik ve bir veya birden fazla modun
modal özelliklerine (periyot, mod şekli, kütle katılım faktörü) yakın
özellikler elde edilmeye çalışılır. Yalıtımlı yapılarda ise kütle
ve rijitlik özelliklerine ek olarak ilgili yönde sadece en etkin ilk
modun modal özelliklerine yakın modal özelliklerin elde edilmesi yeterli
olmaktadır. Bunun nedeni yalıtımlı yapılarda, ilgili yönde, yalıtım
katmanının yerdeğiştirmesi ile ilgili doğal salınım modunun üstyapı
modlarına göre çok daha etkin ve çok yüksek kütle katılım oranına
sahip olmasıdır (\cite{Skinner1993}) (tipik yalıtımlı bir yapıda
bu oran \% 95 mertebelerindedir). Bu makalede, \textcite{Bekin2018-MSThesis}
tarafından yapılan çalışmadan farklı olarak, üstyapı için bir Timoshenko
kiriş modeli kullanılmıştır. Bu modelde, eğilme ve kayma rijitlikleri
ayrı ayrı tanımlanabildiğinden, hastane binalarının modellenmesinde
farklı rijitlik değerlerinin denenmesine imkan verilerek tipik hastane
yapılarının daha gerçekçi modellenmesi hedeflenmiştir.

\begin{figure}
\noindent \centering{}\includegraphics{figures/tikz/typ_baseiso_modelling}\vspace{1cm}
\includegraphics{figures/tikz/common_baseiso_modelling}\caption{\label{fig:typ_common_systems}Tekil ve çoklu yalıtımlı yapı modellemesi.}
\end{figure}
Sismik yalıtımlı tekil ve çoklu sistemlerin hareket denklemleri için
Şekil (\ref{fig:typ_common_systems})'de gösterilen idealleştirilmiş
sistemler kullanılmıştır. Çoklu sistemde $N-$adet üstyapı bulunmaktadır
ve $j.$ yapı katsayısı $n_{j}$'dir. Her kat Timoshenko kirişleri
ile modellenmiştir. Çoklu sistem için hareket denklemi şu şekilde
ifade edilebilir:
\begin{equation}
\mathbf{M}\ddot{\mathbf{x}}(t)+\mathbf{C}\dot{\mathbf{x}}(t)+\mathbf{F}_{\text{s},i}(t)=-\mathbf{M}\mathbf{S}_{1}\ddot{x}_{\text{g}}^{\text{abs}}(t)\label{eq:eq-of-motion}
\end{equation}
\begin{equation}
\mathbf{M}=\begin{bmatrix}\mathbf{M}_{\text{s}} & \mathbf{M}_{\text{s}}\mathbf{R}\\
\mathbf{R}^{\text{T}}\mathbf{M}_{\text{s}} & \mathbf{R}^{\text{T}}\mathbf{M}_{\text{s}}\mathbf{R}+M_{\text{b}}
\end{bmatrix},\quad\mathbf{C}=\begin{bmatrix}\mathbf{C}_{\text{s}} & \mathbf{0}\\
\mathbf{0} & C_{\text{b}}
\end{bmatrix},\quad\mathbf{F}_{\text{s},i}(t)=\mathbf{K}\mathbf{x}(t)+\mathbf{S}_{1}F_{\text{iso}}^{\text{NL}}(t),
\end{equation}
\begin{equation}
\mathbf{K}=\begin{bmatrix}\mathbf{K}_{\text{s}} & \mathbf{0}\\
\mathbf{0} & K_{\text{b}}
\end{bmatrix},\quad\ddot{\mathbf{x}}=\begin{Bmatrix}\ddot{\mathbf{x}}_{\text{s}}^{\text{b}}(t)\\
\ddot{x}_{\text{b}}^{\text{g}}(t)
\end{Bmatrix},\enskip\dot{\mathbf{x}}=\begin{Bmatrix}\dot{\mathbf{x}}_{\text{s}}^{\text{b}}(t)\\
\dot{x}_{\text{b}}^{\text{g}}(t)
\end{Bmatrix},\enskip\mathbf{x}=\begin{Bmatrix}\mathbf{x}_{\text{s}}^{\text{b}}(t)\\
x_{\text{b}}^{\text{g}}(t)
\end{Bmatrix},\quad\mathbf{S}_{\text{1}}=\begin{Bmatrix}\mathbf{0}\\
1
\end{Bmatrix},\quad\mathbf{R}=\begin{Bmatrix}1\\
\vdots\\
1
\end{Bmatrix}
\end{equation}
Burada, $\mathbf{M}_{\text{s}}$, $\mathbf{C}_{\text{s}}$ ve $\mathbf{K}_{\text{s}}$
terimleri sırası ile üstyapı kütle, sönüm ve rijitlik matrislerini,
$M_{\text{b}}$ yalıtım düzlemi kütlesini, $F_{\text{iso}}^{\text{NL}}$
yalıtıcının doğrusal olmayan kesme kuvvetini, $C_{\text{b}}$ ve $K_{\text{b}}$
sırası ile yalıtım seviyesinde bulunması muhtemel ek mekanizmaların
sönümleme ve rijitlik değerlerini ifade etmektedir. Yalıtım düzleminin
zemine göre bağıl yerdeğiştirmesi $x_{\text{b}}^{\text{g}}$ ile,
üstyapının yalıtım düzlemine göre bağıl yer değiştirmesi ise $\mathbf{x}_{\text{s}}^{\text{b}}$
ile gösterilmiştir. $\mathbf{S}_{\text{1}}$ deprem etki vektörünü,
$\mathbf{R}$ ise elemanları ``1'' rakamından oluşan $N_{\textrm{kat}}\textrm{x}1$
boyutunda bir katsayı vektörüdür. Üst yapı kütle, rijitlik ve sönümleme
matrisleri şu şekilde oluşturulmaktadır:
\begin{equation}
\mathbf{M}_{\text{s}}=\begin{bmatrix}\ddots &  & \mathbf{0}\\
 & \mathbf{M}_{\text{s},j}\\
\mathbf{0} &  & \ddots
\end{bmatrix},\quad\mathbf{K}_{\text{s}}=\begin{bmatrix}\ddots &  & \mathbf{0}\\
 & \mathbf{K}_{\text{s},j}\\
\mathbf{0} &  & \ddots
\end{bmatrix},\quad\mathbf{C}_{\text{s}}=\begin{bmatrix}\ddots &  & \mathbf{0}\\
 & \mathbf{C}_{\text{s},j}\\
\mathbf{0} &  & \ddots
\end{bmatrix},\quad j=1,\ldots,N\label{eq:superstr-matrices}
\end{equation}
Bu denklemlerde $\mathbf{M}_{\textsc{s},j}$, $\mathbf{K}_{\textsc{s},j}$
ve $\mathbf{C}_{\textsc{s},j}$ sırası ile $j.$ üstyapı kütlelerinden
oluşan diyagonal kütle matrisi, $j.$ üstyapı için Timoshenko kiriş
matrislerinin matris yöntemleri ile birleştirilmesinden oluşan rijitlik
matrisi ve $j.$ üstyapı için sönümleme matrisidir. Sönümleme matrisi,
farklı yöntemler ile oluşturulabilir. Bu çalışmada, klasik Rayleigh
sönümleme matrisi kullanılmıştır. Yerdeğiştirme vektörü, üstyapı yerdeğiştirmeleri
cinsinden şu şekilde ifade edilmektedir:
\begin{equation}
\mathbf{x}_{\text{s}}^{\text{b}}=\begin{Bmatrix}\vdots\\
\mathbf{x}_{\text{s},j}^{\text{b}}\\
\vdots
\end{Bmatrix},\quad\mathbf{x}_{\text{s},j}^{\text{b}}=\begin{Bmatrix}\vdots\\
x_{j,i}(t)\\
\theta_{j,i}(t)\\
\vdots
\end{Bmatrix},\quad i=1,\ldots n_{j},\quad j=1,\ldots,N\label{eq:disp-vector}
\end{equation}
Sistem matrislerinin elde edilişi ve diğer detayları \textcites{Nagarajaiah1991}{Narasimhan2006}{Erkus2006}{Bekin2018-MSThesis}'de
bulunabilir.

\begin{figure}
\noindent \centering{}\includegraphics{figures/tikz/timo_local}\caption{\label{fig:timo-beam}Timoshenko kirişinin tanımlanmasında kullanılan
yerel koordinat sistemi ve yerel serbestlik dereceleri.}
\end{figure}
Rijitlik matrisinin elde edilmesinde kullanılan ve Şekil (\ref{fig:timo-beam})'de
gösterilen yerel koordinat sistemi ve yerel serbestlik dereceleri
için tanımlanan Timoshenko elemanının rijitlik matrisi, ilgili şekil
fonksiyonları ve kayma alanı katsayısı şu şekilde verilmiştir:
\global\long\def\arraystretch{2.0}%
\begin{equation}
\mathbf{k}_{j,i}=\begin{bmatrix}\dfrac{12}{1+\Phi}\dfrac{EI}{L^{3}} & \dfrac{6}{1+\Phi}\dfrac{EI}{L^{2}} & -\dfrac{12}{1+\Phi}\dfrac{EI}{L^{3}} & \phantom{-}\dfrac{6}{1+\Phi}\dfrac{EI}{L^{2}}\\
 & \dfrac{4+\Phi}{1+\Phi}\dfrac{EI}{L} & -\dfrac{6}{1+\Phi}\dfrac{EI}{L^{2}} & \phantom{-}\dfrac{2-\Phi}{1+\Phi}\dfrac{EI}{L}\\
\textrm{Simetrik} &  & \phantom{-}\dfrac{12}{1+\Phi}\dfrac{EI}{L^{3}} & -\dfrac{6}{1+\Phi}\dfrac{EI}{L^{2}}\\
 &  &  & \phantom{-}\dfrac{4+\Phi}{1+\Phi}\dfrac{EI}{L}
\end{bmatrix},\quad\begin{array}{c}
\Phi=\dfrac{12EI}{G\left(A/\alpha\right)L^{2}},\\
\\
\alpha=\dfrac{A}{I^{2}}\int_{A}\dfrac{Q\left(y\right)^{2}}{b\left(y\right)^{2}}dA
\end{array}\label{eq:timo-stiffness}
\end{equation}
Burada, $E$ elastisite modülünü, $G$ kayma modülünü, $A$ kesit
alanı, $I$ atalet momenti, $L$ eleman boyu, ve $\Phi$ kayma deformasyonlarının
eğilme deformasyonlarına göre bağıl olarak tanımlandığı parametre,
$Q(y)$ kayma gerilmesi ve $b(y)$ 'dir. Timoshenko kirişi ile ilgili
teori ve denklemlerin detayları yapısal analiz kitaplarında bulunabilir
(örnek: \cite{mcguire2015}).

\subsection{Doğrusal Olmayan Analiz Yöntemi}

Bu çalışmada hareket denklemlerinin doğrusal olmayan sayısal integrasyonunda
Newmark-$\beta$ \parencite{Newmark1959method} ve Newton-Raphson
yöntemleri kullanılmıştır. Bu yöntemlerin detayları \textcite{Erkus2004a}'
da verilmiş olup burada kısaca özetlenmiştir. Hareket denkleminin
(Denklem \ref{eq:eq-of-motion}) $t$ ve $t+\Delta t$ zaman adımları
farkını ifade eden her $\Delta t$ zaman aralığı için denklemin artımsal
formu aşağıdaki gibi yazılabilir.
\begin{equation}
\mathbf{M}\Delta\ddot{\mathbf{x}}_{i}+\mathbf{C}\Delta\dot{\mathbf{x}}_{i}+\Delta\mathbf{F}_{\text{s},i}=\Delta\mathbf{P}_{i}\label{eq:eq-of-motion-incr}
\end{equation}
Burada $i$ alt indisi ilgili değerin $t_{i}+\Delta t$ ve $t_{i}$
anlarındaki değerlerinin farkını ifade etmektedir: $\square_{i}=\square(t_{i}+\Delta t)-\square(t_{i})$.
Newmark-$\beta$ yönteminde hız ve yerdeğiştirme vektörleri için şu
kabuller kullanılmaktadır \parencite{Newmark1959method}:
\begin{equation}
\dot{\mathbf{x}}_{i+1}=\dot{\mathbf{x}}_{i}+[(1-\gamma)\Delta t]\ddot{\mathbf{x}}_{i}+(\gamma\Delta t)\ddot{\mathbf{x}}_{i+1},\quad\mathbf{x}_{i+1}=\mathbf{x}_{i}+(\Delta t)\dot{\mathbf{x}}_{i}+[(\dfrac{1}{2}-\beta)(\Delta t)^{2}]\ddot{\mathbf{x}}_{i}+[\beta(\Delta t)^{2}]\ddot{\mathbf{x}}_{i+1}\label{eq:newmark-assumptions}
\end{equation}
Burada, $\beta$ ve $\gamma$, ivmenin $t$ ve $t+\Delta t$ zaman
aralığındaki değişimi hakkında yapılan kabulü belirler; bu çalışmada
sabit ortalama ivme durumuna karşılık gelen $\beta=1/4$ ve $\gamma=1/2$
değerleri kullanılmıştır. Bu kabuller kullanılarak hız ve ivmenin
artımsal halleri şu şekilde ifade edilebilir \parencite{Newmark1959method}:
\begin{equation}
\Delta\dot{\mathbf{x}}_{i}=\dfrac{\gamma}{\beta\Delta t}\Delta\mathbf{x}_{i}-\dfrac{\gamma}{\beta}\dot{\mathbf{x}}_{i}+\Delta t(1-\dfrac{\gamma}{2\beta})\ddot{\mathbf{x}}_{i},\quad\Delta\ddot{\mathbf{x}}_{i}=\dfrac{1}{\beta\Delta t^{2}}\Delta\mathbf{x}_{i}-\dfrac{1}{\beta\Delta t}\dot{\mathbf{x}}_{i}-\dfrac{1}{2\beta}\ddot{\mathbf{x}}_{i}\label{eq:newmark-assumptions-incr}
\end{equation}
Bu denklemler, Denklem \ref{eq:eq-of-motion-incr}'e konursa diferansiyel
hareket denklemi aşağıdaki cebirsel forma dönüşür:
\begin{equation}
\mathbf{A}\Delta\mathbf{x}_{i}+\Delta\mathbf{F}_{\text{s,}i}=\Delta\hat{\mathbf{P}}_{i}\label{eq:eq-of-motion-incr-algebric}
\end{equation}
\begin{equation}
\mathbf{A}=\dfrac{1}{\beta\Delta t^{2}}\mathbf{M}+\dfrac{\gamma}{\beta\Delta t}\mathbf{C},\quad\Delta\hat{\mathbf{P}}_{i}=\Delta\mathbf{P}_{i}+\left(\dfrac{1}{\beta\Delta t}\mathbf{M}+\dfrac{\gamma}{\beta}\mathbf{C}\right)\dot{\mathbf{x}}_{i}+\left[\dfrac{1}{2\beta}\mathbf{M}+\Delta t\left(\dfrac{\gamma}{2\beta}-1\right)\mathbf{C}\right]\ddot{\mathbf{x}}_{i}\label{eq:eq-of-motion-incr-algebric-matrices}
\end{equation}
Burada $\Delta\mathbf{F}_{\text{s,}i}$ terimi $\Delta t$ zaman aralığında
izolatör ve üstyapının ürettiği içsel kuvvetlerin toplamını ifade
etmektedir ve $\Delta\mathbf{x}_{i}$ yer değiştirmesine bağlıdır.
Bundan dolayı \ref{eq:eq-of-motion-incr-algebric} ile verilen hareket
denklemi doğrudan çözülemez. 
\begin{figure}[h]
\noindent \centering{}\includegraphics{figures/NewtonRaphson}\caption{\label{Newton-Raphson}Newton-Raphson yineleme adımları.}
\end{figure}
Newton-Raphson yöntemi bu denklemin çözümünde kullanılan yinelemeli
bir yöntemdir. Bu yöntemde, her $j.$ yinelemede önce içsel kuvvetler
hakkında teğetsel rijitlik matrisi $K_{\text{T},i}^{j}$ kullanılarak
bir kabul yapılır ve bu kabule denk gelen yerdeğiştirmeler Denklem
\ref{eq:eq-of-motion-incr-algebric} kullanılarak hesaplanır:
\begin{equation}
\Delta\mathbf{F}_{\text{s,}i}^{\text{kabul,}j}=K_{\text{T,}i}^{j}\Delta\mathbf{x}_{i}^{j},\quad\Delta\mathbf{x}_{i}^{j}=\Delta\hat{\mathbf{P}}_{i}(\mathbf{A}+\mathbf{K}_{\text{T,}i}^{j})^{-1}\label{eq:assumed-int-force-disp}
\end{equation}
Bu yerdeğiştirmeye denk gelen ve yapının ürettiği içsel kuvvetler
$\Delta\mathbf{F}_{\text{s,}i}^{\text{iç,}j}$ bünye denklemlerinden
elde edilir. Kabul edilen iç kuvvetler ile yapının ürettiği iç kuvvetler
arasındaki fark \textit{dengelenmemiş kuvvet} $\mathbf{F}_{\text{s,}i}^{\text{dk,}j}$
olarak adlandırılır. Her artımda denge noktasını yakalayabilmek için
dengelenmemiş kuvvetler nedeni ile oluşacak ek yerdeğiştirmeler her
yinelemede hesaplanmalıdır. Bunun için bir yinelemede elde edilen
dengelenmemiş kuvvetler bir sonraki yinelemede yapıya dış kuvvet olarak
etkitilir:
\begin{equation}
\mathbf{F}_{\text{s,}i}^{\text{dk,}j}=\Delta\mathbf{F}_{\text{s,}i}^{\text{kabul,}j}-\Delta\mathbf{F}_{\text{s,}i}^{\text{iç,}j},\quad\mathbf{A}\Delta\mathbf{x}_{i}^{j+1}+\Delta\mathbf{F}_{\text{s,}i}^{j+1}=\mathbf{F}_{\text{s,}i}^{\text{dk,}j}\label{eq:unbalanced-force}
\end{equation}
Newton-Raphson yinelemesinin $j+1$ adımı için yazılan hareket denklemi
$j$ adımına benzer şekilde çözülerek, her yinelemede yeni değiştirme
elde edilir. Yinelemeler, belli bir hata kriteri sağlanana kadar devam
eder ve yerdeğiştirmelerin toplamı ilgili artımdaki yerdeğiştirme
vektörünü verir:
\begin{equation}
\Delta\mathbf{x}_{i}=\sum\limits _{j\text{=1}}^{n}\Delta\mathbf{x}_{i}^{j}\label{eq:newton-raphson-total-disp}
\end{equation}


\subsection{Eşdeğer TSDS Yöntemi – Tekil Yalıtımlı Yapılar}

Bu yöntem yinelemeli bir yöntemdir. Mühendislik uygulamalarında her
biri farklı sıra ile ya da farklı şekilde uygulanıyor olsa da, her
yinelemenin üç ana aşamadan oluştuğu kabul edilebilir: (a) üstyapı
ve yalıtıcı tasarımı, (b) yalıtıcı tepkilerinin hesaplanması. (c)
üstyapı kat kuvvetlerinin belirlenmesi. Bu bölümde ilk önce bu aşamalar
aşağıda sırası ile açıklanmıştır. Daha sonra örnek olması açısından
yöntemin mühendislik uygulamalarında takip edilen bir hali özetlenmiştir.
Son olarak, yöntem hakkında yorum ve öneriler sunulmuştur.

\subsubsection*{Üstyapı ve Yalıtıcı Tasarımı}

Bu aşamada amaç, üst yapının boyutlandırılması ve detaylandırılması,
yalıtıcıların boyutlandırılması, özelliklerinin belirlenmesi ve detaylandırılması,
ve belirlenen boyut ve özellikler için çevrimsel davranışlarının belirlenmesidir.
Üst yapı tasarımı için bir önceki yinelemede elde üstyapı kuvvetleri,
yalıtıcı tasarımı için ise bir önceki yinelemeden elde edilen yalıtım
seviyesi yerdeğiştirmeleri ve eksenel kuvvetler kullanılır.

İlk yinelemede üstyapı kuvvetleri ve yalıtıcı tepkileri henüz bilinmediğinden,
daha çok tecrübeye dayalı ön tasarımlar gerçekleştirilir. Örnek olarak
üst yapı tasarlanırken, yönetmeliklerde sadece düşey yükler için verilen
yük kombinasyonları kullanılabilir. Yalıtıcı için ise kapsamlı bir
ürün tasarımı yapmak yerine, üstyapı periyoduna, kütlesine bağlı olarak
bir çiftdoğrusal çevrimsel davranış kabulü (ilk ve akma sonrası rijitlik
değerer, akma noktası kuvvet ve yerdeğiştirmeler) yapmak yeterli olmaktadır.

\subsubsection*{Yalıtıcı Tepkilerinin Hesaplanması}

\begin{figure}
\noindent \centering{}\includegraphics{figures/tikz/tsds_modeling}\caption{\label{fig:tsds-modelling}Yalıtımlı yapının TSDS'e indirgenmesi}
\end{figure}
Bu aşamada amaç verilen bir deprem spektrumu için yalıtıcı yerdeğiştirmelerini
ve yalıtım seviyesi kesme kuvvetlerini hesaplamaktır. Bu tepkilerinin
hesabının verilen bir spektrum altında yapılabilmesi için yalıtımlı
yapının ilk önce bir eşdeğer doğrusal olmayan TSDS'e indirgenmesi
gerekir (Şekil \ref{fig:tsds-modelling}). Bu indirgemede iki temel
kabul yapılır. İlk kabul, üst yapı katlarının yalıtım diyaframına
göre göreli bir yerdeğiştirme yapmadığı, üstyapının tümden rijit bir
kütle olarak davrandığıdır. Bu kabul aynı zamanda üstyapıda olulaşacak
salınımların, yalıtıcı davranışını etkilemediğini anlamına gelmektedir.
Üstyapı salınımm periyotlarının, yalıtım seviyesinin etkin salınım
periyotlarından ayrık olması durumunda bu kabul daha gerçekçi olmaktadır.

Yapılan diğer kabul, yalıtım seviyesindeki tüm yalıtıcıların, doğrusal
olmayan çiftdoğrusal davranış gösteren toplu bir yalıtıcı olarak modellenmesidir.
Bu noktada, tipik çiftdoğrusal davranış yerine, yalıtıcıların davranışını
daha doğru gösteren çevrimsel modeller de kullanılabilir. Bu toplu
yalıtıcı modelinin doğrusal olmayan davranışı, tüm yalıtıcıların doğrusal
olmayan davranışlarının üstdüşümü ile elde edilir. Örnek olarak toplu
yalıtıcının akma kuvveti, her bir yalıtıcının akma kuvvetlerinin toplamı
olarak bulunabilir. Yalıtım katmanında ek bir sönümleme mekanizması
kullanılmıyor ise, bu sistemde bir viskoz sönümleme elemanı bulunmaz.

\begin{figure}
\noindent \centering{}\includegraphics{figures/tikz/hyster-viscous}\caption{\label{fig:viscous-model}Doğrusal olmayan davranışın doğrusal yay
ve viskoz sönümlemeye indirgenmesi}
\end{figure}
Doğrusal olmayan TSDS, ikinci defa basitleştirilerek doğrusal yaylı
ve viskoz sönümlemeli bir yapıya indirgenir (Şekil \ref{fig:viscous-model}).
Bu yaklaşımda iki kabul vardır. Bunlardan ilki TSDS'in verilen bir
maksimum yerdeğiştirme ve maksimum kuvvet değerlerinde çevrimsel bir
salınım yaptığıdır. İkinci kabul ise, çevrimsel davranış ile sönümlenen
enerjinin viskoz bir mekanizma ile sönümlendiğidir. Bu durumda, eşdeğer
doğrusal yayın rijitiliği doğrusal olmayan yayın maksimum yerdeğiştirme
ve maksimum kuvvete denk gelen sekant rijitliği olmaktadır. Viskoz
sönümleyici maksimum yerdeğiştirme değerinde salınım yapmaktadır.
Bu durumda viskoz sönümleyicinin kritik sönümleme oranı, şu şekilde
hesaplanabilir:
\begin{equation}
\xi_{\text{eş}}=\dfrac{1}{4\pi}\dfrac{E_{\text{D}}}{E_{\text{S}}}\label{eq:effective-damping}
\end{equation}
Burada, $E_{\text{D}}$ doğrusal olmayan yayın bir çevrimde sönümlediği
enerji ve $E_{\text{S}}$ ise doğrusal yayda ilgili yerdeğiştirme
anında oluşan birim şekildeğiştirme enerjisidir. Bu iki değerde, verilen
bir doğrusal olmayan çevrimsel davranış için hesaplanabilir. Örnek
olarak Şekil \ref{fig:typ-bilinear}'de gösterilen çiftdoğrusal çevrim
için $E_{\text{D}}=4Q_{\text{y}}(x_{\text{b}}^{\text{max}}-x_{\text{y}})$
ve $E_{\text{S}}=k_{\text{eş}}(x_{\text{b}}^{\text{max}})^{2}/2$
'dir.
\begin{figure}[h]
\noindent \centering{}\includegraphics{figures/Bilinear}\caption{\label{fig:typ-bilinear}İzolatörlerin çift doğrusal histeretik eleman
modeli.}
\end{figure}

Görüldüğü üzere, TSDS'de yukarıda özetlenen indirgemeyi yapabilmek
için, bir başka deyişle eşdeğer rijitlik ve sönümleme değerini bulabilmek
için, verilen spektrum altında oluşacak yerdeğiştirme değerinin bilinmesi
gerekir. Verilen spektrum altında yerdeğiştirme hesabının yapılabilmesi
için ise TSDS'in periyodu ve sönümlemenin bilinmesi gerekir ki, periyot
kütle ve eşdeğer rijitliğe bağlıdır. Bundan dolayı, yerdeğiştirme
ve eşdeğer rijitlik ve eşdeğer sönümleme değerlerinin hesaplanması
birbirlerine bağlı olup, hesaplama yinelemeli bir yöntem ile yapılır
(Şekil \ref{fig:tsds-dependency}). Bu noktada vurgulanması gereken
bu yinelemenin eşdeğer TSDS yönteminin ana yinelemelerden içinde bulunan
alt yinelemeler olduğudur.
\begin{figure}
\noindent \centering{}\includegraphics{figures/IterativeAnalysis}\caption{\label{fig:tsds-dependency}Eşdeğer rijitlik, eşdeğer sönümleme ve
yerdeğiştirme hesaplarının birbirine bağlılığı. }
\end{figure}

Eşdeğer doğrusal TSDS'in yerdeğiştirmelerinin hesabı için yapılan
yinelemeler şu şekildedir: İlk önce eşdeğer doğrusal TSDS'in eşdeğer
periyodu ve eşdeğer sönümleme oranı hakkında tecrübeye dayalı bir
kabul yapılır. Bu periyod için spektral ivme verilen spektrumdan elde
edilir. Eğer verilen spketrum \%5 kritik sönümleme oranı için verilmiş
ise, bu spektrum eşdeğer sönüm değeri için genelde yönetmeliklerde
verilen katsayılar ile güncellenir. Elde edilen spektrum değerine
denk gelen yerdeğiştirme hesaplanır. Daha sonra, bu yerdeğiştirmeye
denk gelen yeni bir eşdeğer periyod ve eşdeğer sönüm değeri elde edilir.
Bir sonraki yineleme bu yeni değerler için tekrarlanır.

\subsubsection*{Üstyapı Kat Kuvvetlerinin Belirlenmesi}

\begin{figure}
\noindent \centering{}\includegraphics{figures/tikz/story_shear}\caption{\label{fig:story-shears-def}Kat ağırlıkları, eşdeğer kat ve kesme
kuvvetleri ve oranlarının tanımlanması}
\end{figure}
Eşdeğer TSDS yönteminde, yalıtım seviyesi yerdeğiştirme ve kesme kuvvetini
bulduktan sonraki aşamada üstyapı eşdeğer statik deprem kuvvetlerinin
bulunur. Bunun için akademik ve yönetmelik literatüründe çok farklı
yaklaşımlar mevcut olup bu bölümde en fazla bilinenleri kısaca açıklanmıştır.
Bu noktada, kat kesme kuvvetlerinin, $V_{i}$, ve kat kesme oranlarının,
$C_{i}$, tanımlanması uygun olmaktadır (Şekil \ref{fig:story-shears-def}):
\begin{equation}
C_{j}=\dfrac{V_{j}}{W_{j}},\quad W_{j}=\sum_{i=j}^{k}w_{i},\quad j:\text{b},1,\ldots,k\label{eq:story-force-ratio}
\end{equation}
Burada $w_{i}$, $i$ numaralı katın sismik ağırlığı, $k$, kat sayısı,
$\text{b}$, yalıtım döşemesinin indisi, $W_{j}$, $j$ katı ve üstündeki
katların sismik ağırlıklarının toplamı ve $V_{j}$, $j$ seviyesi
kat kesme kuvvetidir. Yalıtım seviyesi için bu oran şu hali almaktadır:
\begin{equation}
C_{\text{b}}=\dfrac{V_{\text{b}}}{W_{\text{b}}+W_{\text{T}}},\quad W_{\text{T}}=\sum_{i=1}^{k}w_{i}\label{eq:iso-force-ratio}
\end{equation}
Burada, $W_{\text{T}}$, üst yapı katlarının yalıtım döşemesi dahil
olmadan toplam sismik ağırlığı, $W_{\text{b}}$, yalıtım döşemesi
sismik ağırlığı ve $V_{\text{b}}$ yalıtım seviyesi kesme kuvvetidir.

\begin{figure}
\noindent \centering{}\includegraphics{figures/tikz/equiv-loads}\caption{\label{fig:story-shears-methods} Farklı yöntemler ile eşdeğer kat
kuvvetlerinin bulunması}
\end{figure}
İlk yaklaşımda, yalıtım seviyesi kesme kuvvetleri, $V_{\text{b}}$,
yalıtım ve kat döşemeleri olmak üzere tüm döşemelere ters üçgen tabir
edilen bir düzen ile dağıtılır:
\begin{equation}
F_{j}=\dfrac{V_{\text{b}}w_{j}h_{j}}{\sum_{i=\text{b}}^{k}w_{i}h_{i}},\quad j:\text{b},1,\ldots,k\label{eq:ASCE7-10-Fi}
\end{equation}
 Bu yaklaşım eski \textcites{ASCE7-05}{ASCE7-10}{ASCE41-06} standartlarında
ve yeni \textcite{TBDY2018}'de kullanılmaktadır.

İkinci yaklaşımda, yalıtım seviyesi kesme kuvvetlerinden ilk kat seviyesi
kesme kuvveti ve buna ağlı olarak yalıtım döşemesine uygulanan kuvvet
bulunur. Daha sonra ilk kat seviyesi kesme kuvveti ters üçgen dağılım
benzeri bir dağılımla üst katlara dağıtılır. \textcites{ASCE7-16}{ASCE41-13}{ASCE41-17}'deki
yaklaşım bu şekildedir: Örnek olması açısından burada \textcite{RyanYork2008}
çalışmasına dayanan \textcite{ASCE7-16}'deki formüller verilmiştir.
Birince kat kesme ve yalıtım döşemesi kat kuvvetleri, yalıtım seviyesi
kesme kuvveti kullanılarak şu şekilde hesaplanmaktadır:
\begin{equation}
V_{\textrm{1}}=V_{\textrm{b}}\left(\dfrac{W_{\textrm{T}}}{W_{\text{b}}+W_{\text{T}}}\right)^{\left(1-2.5\xi\right)},\quad F_{\text{b}}=V_{\text{b}}-V_{1}\label{eq:ASCE-17-V1}
\end{equation}
Burada $\xi$, ilgili eşdeğer TSDS'in eşdeğer viskoz sönümleme oranıdır.
$2.5$ faktörü çok sert doğrusalsızlık gösteren yalıtıcılar için $3.5$
olmaktadır. KAt kuvvetleri ise şu şekilde hesaplanmaktadır:
\begin{equation}
F_{j}=\dfrac{V_{\text{b}}w_{j}h_{j}^{k}}{\sum_{i=2}^{k}w_{i}h_{i}^{2}},\quad k=14\xi T_{\text{fb}},\quad j:2,\ldots,k\label{eq:ASCE7-17-Fi}
\end{equation}
Burada $T_{\text{fb}},$ üstyapının yalıtımsız halinin periyodudur.

Üçüncü yaklaşımda kat kuvvetleri, yalıtımlı yapıların etkin mod şekilleri
göz önüne alınarak dağıtım yapılmaktadır \parencites{LeeHongKim2001}{TsaiChenChiang2003}:
\begin{equation}
F_{j}=V_{\text{b}}\dfrac{w_{j}(1+rh_{j})}{\sum_{i=\text{b}}^{k}w_{i}(1+rh_{i})},\quad r=\dfrac{\varepsilon}{H_{\text{e}}},\quad\varepsilon=(\dfrac{T_{\text{eş}}}{T_{\text{fb}}})^{2}\label{eq:Lee-Fi}
\end{equation}
Burada $T_{\text{eş}}$ ve $T_{\text{fb}}$ sırası ile eşdeğer TSDS'in
ve üstyapının yalıtımsız halinin periyotları ve $H_{\text{e}}$, efektif
yüksekliktir.

Son yaklaşım pratik bir ön tasarım yaklaşımıdır. Bu yaklaşımda, ilk
kat kesme kuvveti hakkında bir kabul yapılır. Örnek olarak yalıtım
döşeme ve üstyapı kat ağırlıklarının benzer olduğu yapılar için birinci
kat kesme kuvvet oranının yalıtım seviyesi kesme kuvvet oranına eşit
olduğu kabul edilebilir: $C_{1}=C_{\text{b}}.$ Bu kabul ile ilk kattaki
kat kesme kuvvetleri hesaplanabilir, $V_{1}=C_{1}W_{\text{T}}$ ve
bu kuvvet katlara ters üçgen düzeninde dağıtılabilir. Eğer yalıtım
döşeme ağırlığı kat kütlelerinden oldukça farklı ise üstyapı rijitliği
düşükse, Denklem \ref{eq:ASCE-17-V1} kullanılarak da bu oran hesaplanabilir.

\subsubsection*{Eşdeğer TSDS'in Mühendislikte Uygulanması}

Eşdeğer TSDS yönteminin mühendislikteki uygulamaları genelde yukarıda
açıklanan yinelemeli aşamalardan sıra yönünden farklılık göstermektedir.
Uygulamada genellikle ilk önce eşdeğer periyod ve sönümleme hakkında
tecrübeye dayalı hedef değerler ana kriter olarak belirlenir. Bunun
nedeni, yalıtımlı yapılarda olası eşdeğer periyod ve sönümleme değerlerinin
çok geniş bir aralığa sahip olmasıdır. Bundan dolaayı, bu değerler
ana kriter olarak en başta belirlenmez ise yukarıda açıklanan yinelemeli
aşamaların uygulanması mümkün olmaz. Hedef eşdeğer periyot ve sönümleme
ana kriter olarak belirlendikten sonra üst yapı, bu hedef periyot
ve sönümlemeye denk gelen taban kesme kuvvetlerine göre yalıtıcı tasarımından
bağımsız olarak tasarlanır. Üst yapı tasarımı sırasında taban kesme
kuvvetlerinin azaltılması gerektiği anlaşılırsa, hedef periyot ve
sönümleme değerleri yine tecrübeye bağlı olarak artırılabilir. Yalıtıcılar
için ise, üstyapı tasarımına paralel olarak hedef periyot ve sönümleme
değerlerini sağlayan yalıtıcı tipi, boyutları ve özellikleri belirlenir.
Yalıtıcı tasarımı üstyapı tasarımına yalıtıcı tipine göre bağlı olabilir
ya da ondan tamamen bağımsız olabilir. Bu durum elastomer ve sürtünme
tabanlı yalıtıcılar için ayrı ayrı açıklanmıştır.

Elastomer tabanlı yalıtıcılar için, yalıtım sisteminin eşdeğer periyodu
üstyapı kütlesine, yalıtıcı genişliği ise burkulma tasarımı nedeni
ile eksenel yük ve yerdeğiştirme taleplerine bağlıdır. Yalıtıcı yarıçapı
ise kayma rijitliğine ve buna bağlı olarak eşdeğer periyodu etkiler.
Bundan dolayı, elastomer tabanlı yalıtıcılar ile her istenen periyot
ve sönümleme oranını sağlayan yalıtıcı birimi tasarlamak mümkün olmayabilir
ve bu tasarım üstyapı tasarımına doğrusan bağlıdır. Özellikle uzun
periyot değerlerini elastomar tabanlı yalıtıcılar ile sağlamak zor
olduğundan, bazı durumlar elastomer yalıtıcılar yerine düz yüzeyli
kayıcı mesnet kullanılarak periyot artımı sağlanabilir.

Sürtünme tabanlı yalıtıcılarda, yalıtım sistemi eşdeğer periyodu üstyapı
kütlesinden bağımsızdır ve yerdeğiştirme talebi yalıtıcı hareket çapını,
eksenel yükler ise birim içerisindeki çelik bölümlerin çapını belirler.
Ayrıca sönümleme sürtünme katsayısına bağlı olduğundan istenilen bir
eşdeğer sönümleme oranı, sürtünme etkisini yaratan malzemenin özelliklerine
bağlı olarak istenildiği gibi belirlenebilir. Bundan dolayı, sürtünme
tabanlı sönümleyiciler ile istenilen hedef periyot ve sönümleme üstyapı
tasarımından tamamen bağımsız olarak gerçekleştirilebilir.

Görüldüğü üzere, yalıtıcı tasarımı her iki tip yalıtıcılar ile ilgili
olarak, tecrübe ve yalıtıcı davranış ve tasarım bilgisi gerektirmektedir.
Bu noktada, bazı tasarım kılavuzları, yönetmelikler kullanılabilir
(örnek: \cite{AASHTO2010-BaseIso,Constantinou2011}) ya da üreticilerden
boyutlandırma konusunda destek alınarak hem veri yokluğunda ilk ön
tasarım, hem sonraki yinelemelerde kapsamlı tasarım yapılabilir.

\subsubsection*{Eşdeğer TSDS Yöntemi Hakkında Görüş ve Öneriler}

Yukarıda anlatıldığı üzere eşdeğer TSDS yöntemi bünyesinden birçok
kabul barındırmaktadır. Yalıtım tepkilerinin hesaplanmasında üstyapı
rijitliği ihmal edilmektedir. Yalıtım doğrusalsızlığı, doğrusal bir
yay ve viskoz sönümlemeye indirgenmiştir. Ayrıca, farklı yönetmeliklerde
verilen spektrum azaltma katsayısı birbirinden oldukça farklı olabilir
\parencite{Erkus2017a}. Yalıtım periyotları mertebelerinde spektrum
ile yerdeğiştirme hesabı çok hassaslaşmaktadır. Doğrusal olmayan zaman-analiz
yöntemleri ile bir karşılaştırma yapılacaksa kullanılan deprem kayıtlarının
verilen bir spektruma uygun hale getirilmesi ile ilgili yöntemler
de tartışmaya açıktır. Tüm bu noktalara rağmen, eşdeğer TSDS yöntemi
ile elde edilen yalıtım tepkilerinin genelde güvenli tarafta kaldığı
bilinmektedir.

Üstyapı kuvvetleri hesabında kullanılan yöntemler de yaklaşık yöntemlerdir.
Bundan dolayı bu yöntemler nedeni ile elde edilen kuvvetlerin ne kadar
doğru olduğu her zaman tartışmaya açıktır. Örnek olarak, birinci yöntem
hata payı en yüksek yöntem olarak bilinmekte, yapı özellikleri az
katlı standart yalıtımlı yapı tipinden farklılaştıkça bu hata artmaktadır
ve üstyapı kat kuvvetlerinin dağılımı gerçekçi olmamaktadır \parencites{LeeHongKim2001}{TsaiChenChiang2003}{RyanYork2008}{Cardone2009}{Erkus2017a}.
Normalde, yalıtımsız yapılarda kullanılan bu yöntemin yalıtımlı yapılara
uyarlanmasının nedeni, yüksek modların etkin olabileceği üstyapılarda
(örnek: katsayısının fazla olduğu üstyapılar) bu etkilerin göz önüne
alınması olarak açıklanmaktadır \parencite{FEMA450Part2}. Bu yöntemde,
eğer yalıtım seviyesi orta kotu, yalıtım döşeme kotuna çok yakınsa,
yalıtım döşemesine gelen kuvvet sıfıra yakın olacaktır. Ancak bu durumda
bu yöntem yalıtım döşemesi ağırlığının kat ağırlıklarına göre çok
yüksek olduğu yapı tipleri için gerçekçi olmayan sonuçlar verecektir
(örnek: \cite{Erkus2017a}). İkinci ve üçüncü yöntemlerin daha gerçekçi
sonuçlar verdiği ilgili literatürde gösterilmiştir. Üçüncü yöntem
ise daha hızlı hesap yapmak için kullanılmaktadır ve dikkatli kullanılması
durumunda, ikinci ve üçüncü yöntem gibi daha doğru sonuçlar verebilir.

Yukarıda açıklanan yöntemler her ne kadar kendi kapsamları dahilinde
iyi sonuçlar verebiliyor olsa da, gerçek mühendislik uygulamalarında
tüm bu yöntemler yaklaşık yöntemlerdir ve özellikli ve maliyetli yalıtımlı
yapılarda, gerçek davranışın anlaşılması her zaman önceliklidir. Gerçek
davranış ise en etkin doğrusal olmayana zaman-tanım analizleri ile
anlaşılmalı, ve tasarım teyit edilmeli ve yönetmeliklerin yalıtımlı
ya da yalıtımsız tüm yapı tipleri için öngördüğü en temel şart olan
D1 deprem seviyesinde en az göçme öncesi performans kriterinin sağlandığı
ispatlanmalıdır. Ancak bu ileri analizlerin yapılabilmesi için tasarımın
(boyutlar ve detaylar) bitmiş olması gerekir ki tasarımın başında
bu mümkün değildir. Bundan dolayı, çok basit yapılar dışında mühendisi
tasarım için bu ileri analizleri kullanması mümkün değildir ve eşdeğer
TSDS yada eşdeğer mod birleştirme yöntemlerinin kullanılması zorunludur.
Bu noktada yalıtıcı ve üst yapı tasarımları yapılırken eşdeğer TSDS
yönteminin kullanılması nedeni ile oluşan riskler iyi anlaşılmalıdır.

Yalıtıcı tasarımı açısından şu değerlendirme yapılabilir. Yalıtımlı
yapılarda inşaat planlaması düşünüldüğünde, yalıtıcıların, ileri analizler
yapılmadan siparişlerinin verilmesi gerekmektedir. Bundan dolayı tasarım
için ileri analizlerin yapılamamasına ek olarak, yalıtıcıların sipariş
ve inşasından önce de ileri analizler yapılamaz. Bu olumsuzluklara
rağmen, eşdeğer TSDS yöntemi ile elde edilen yalıtıcı tepkilerinin
genelde güvenli tarafta sonuçlar verdiği bilinmektedir. Ayrıca, tasarımda
her zaman bir miktar marj bulunmaktadır. Yalıtımlı yapı tecrübesi
olan mühendis, gerçek yalıtıcı davranışını iyi tahmin edebilir. Ancak,
bir yalıtımlı yapı projesinde, olumlu yönler ne kadar baskın olsa
da, mühendislik etik ve yasal uygulama esasları açısından yalıtıcı
tasarımı mutlaka ileri doğrusal olmayan analizler ile, yalıtıcılar
inşa edilmiş olsa bile, mutlaka kontrol edilmeli ve yalıtıcıların
ve üstyapı ve altyapı bağlantılarının (betonarme elemanlar için ankrajlar,
çelik elemanlar için kaynak ve bulonlar) özellikle D1 depreminde hasar
almayacağı gelecek için raporlanarak kayıt altına alınmalıdır.

Eşdeğer TSDS yönteminde hesaplanan kat kuvvetleri ile ilgili de şu
riskler tartışılabilir. Eşdeğer yöntem ile yapılan üstyapı tasarımı
ise her zaman tartışmaya açıktır ve tüm yöntemler ya yaklaşık ya da
ampirik yöntemlerdir. Ayrıca yönetmelikler üstyapının tasarımını D2
deprem seviyesinde yaptırır ve düşük sünekliğe sahip detaylar ve bir
deprem azaltma katsayısının kullanılmasına izin verir. Bu durumda,
yapının sağlaması gereken en temel kriter olan D1 seviyesinde en az
göçme öncesi performansı tasarım sırasında kontrol edilmez. Bu olumsuzluklara
rağmen yalıtımlı yapılarda üstyapının doğrusal ya da doğrusal yakın
kalması genel yalıtımlı yapı teorisine uyumluluk açısından genel olarak
uygulanmaya çalışılan bir kuraldır. D2 deprem seviyesinde kullanılan
azaltma katsayısı ise yalıtımsız yapı için izin verilen katsayının
en fazla $3/8$'i olabilir ve süneklikten çok bu katsayı içerisinde
barınan güç fazlasını ifade eder. Bundan dolayı üstyapının D1 deprem
seviyesinde doğrusala yakın davranış gösterme olasılığı yalıtımsız
yapılara göre çok daha yüksektir. Bazı durumlarda D2 seviyesinde doğrusal
olmayan zaman-tanım alanında analizleri ile tasarım kontrol edilir.
Tüm bu olumlu yönler yapının D1 deprem seviyesinde beklenen bir davranış
göstermesini sağlayacaktır. Ancak mühendislik uygulama esasları açısından,
yalıtıcı tasarımında olduğu gibi, üstyapı tasarımının D1 deprem seviyesinde
ileri doğrusal olmayan analizler ile kontrol edilip yapının beklenen
performansa sahip olacağı gösterilmeli ve gelecek için raporlanarak
kayıt altına alınmalıdır.

Görüldüğü üzere, hem yalıtıcı tasarımı hem de üstyapı tasarımında
eşdeğer TSDS yönteminin olumlu yönleri olsa da, yöntemin birçok kabul
içermesi ve yaklaşık olması nedeni ile tasarım D1 seviyesi deprem
için mutlaka kontrol edilmelidir. ASCE 7 gibi standartlarda, yalıtımlı
yapıların yaygınlaştırılması maksadı ile belli şartları sağlayan çok
tipik yapılar için bu kontrol yapılmamakla beraber diğer yalıtımlı
yapılar için D1 deprem seviyesinde hem yalıtım hem de üstyapı tasarımı
kontrol edilmesi zorunludur. \textcite{TBDY2018}'de ise tüm yalıtımlı
yapılar için bu kontrolün yalıtım tasarımı için yapılması gerekirken,
üstyapı için yapılma şartı bulunmamaktadır.

\subsection{Eşdeğer TSDS Yöntemi – Ortak Yalıtım Düzlemindeki Yapılar}

Eşdeğer TSDS yönteminin ortak yalıtım düzleminde bulunan yapılara
tüm aşamaları ile uygulanması mümkün değildir. Yukarıda açıklanan
aşamalardan ikinci aşamanın çoklu yapı sistemine uygulanabileceği
tartışılabilir. Bu aşama için, Şekiller\ref{fig:typ_vs_common} ve
\ref{fig:typ_common_systems}'de gösterilen çoklu yapıların yalıtım
döşemesi ve döşeme üstünde kalan tüm yapı kütlesi eşdeğer TSDS'in
kütlesi olarak kabul edilebilir. Bu durumda, verilen bir spektrum
için yalıtım seviyesi yerdeğiştirme ve kesme kuvvetleri ve eşdeğer
periyot ve sönümleme değerleri hesaplanabilir. Ancak bu aşamada tüm
üstyapının kendi bünyesinde salınım olmayan rijit bir kütle olarak
kabul edilmesi, üstyapıların salınım periyotlarının yalıtım periyodu
ile etkileşmemesine bağlıdır. Ancak, özellikle hastane projelerinde
üstyapılar birbirlerinden çok farklıdır ve bazılarının periyotları
eşdeğer yalıtım periyoduna yaklaşabilir. Ayrıca, üstyapıların da kendi
aralarında etkileşimleri olabilir be etkileşimler tüm sistemin davranışını
etkileyebilir. Bu olasıklara rağmen eşdeğer TSDS modellemesi, ön tasarım
kullanılabilir.

Tekil yalıtımlı yapılar için açıklanan ve üstyapı kuvvetlerinin bulunduğu
eşdeğer TSDS yönteminin üçüncü aşaması ve buna bağlı olarak birinci
aşamadaki üstyapı tasarımı doğrudan ortak yalıtımlı yapı sistemine
uygulanamayacağı açıktır. Bu noktada, mühendis tamamen mühendislik
öngörüsü ile bazı hesaplar yapabilir. Örnek olarak Üçüncü aşamadaki
dördüncü yöntem kullanılarak üstyapı kuvvetleri bulunabilir. Ancak
zaten tekil yalıtımlı yapılarda bile doğruluğu her zaman tartışma
konusu olan üstyapı kuvvetlerinin hesabı, çoklu yapı sisteminde yapının
son tasarımı için kullanılamayacağı açıktır.

Bu noktada yalıtımlı yapıların tüm tasarım süreci hatırlanmaladır.
Eşdeğer TSDS yöntemi ya da mod birleştirme yöntemi, yapı tasarımının
yapılmasına yardımcı olan araçlar olup, tüm yalıtımlı yapı doğrusal
olmayan zaman-tanım analizlerine tabi tutularak, yaklaşık yöntemler
ile elde edilen tasarım teyit edilmeledir. Aynı durum ortak yalıtımlı
yapılar için de geçerlidir. Yaklaşık ve hata oranı yüksek te olsa
her türlü tasarım yaklaşımı, doğrusal olmayana anazliler ile teyit
edildiği sürece kullanılabilir. Çoklu yapılarda bu tip analizlerin
oldukça zaman alacağı ve uygulanmasının zor olacağı tahmin edilebilir.
Ancak, günümüz bilgisayar teknolojileri ile ve üstyapı modellerinde
doğrusal ve rijit diyafram yaklaşımları kullanarak bu analizlerin
etkin yapılması da mümkündür.

Ülkemizdeki ortak yalıtım düzlemine sahip yalıtımlı hastane yapılarında
eşdeğer TSDS yönteminin kullanıldığı bilinmektedir. Bu yöntem ile
elde edilen yalıtım seviyesi kesme kuvvetinin üstyapıların kütlelerine
oranlayarak üstyapılara tabana kesme kuvveti olarak dağıtıldığı ya
da yapıların ayrı ayrı yalıtımlı olarak modellendiği ve bu modellerin
mod birleştirme analiz ile tasarlandığı tahmin edilmektedir. Tüm üstyapıların
olduğu bir modelin kapsamlı ve gerçekçi doğrusal olmayan zaman-tanım
analizlerine tabi tutularak tüm üstyapı tasarımlarının teyit edildiği
herhangi bir literatürde bildirilmemiştir.

\subsubsection{Notasyonlar}

Bu çalışmada izolasyon seviyesi kesme kuvveti katsayısı $C_{\textrm{iso}}$,
ve üstyapı taban kesme kuvveti katsayısı $C_{\textrm{s}}$ aşağıdaki
biçimde tanımlanmıştır.

\begin{equation}
C_{\textrm{iso}}=\dfrac{F_{\textrm{iso}}}{W}\quad C_{\textrm{s}}=\dfrac{F_{\textrm{s}}}{W_{\textrm{s}}}
\end{equation}

Burada $F_{\textrm{iso}}$ izolasyon seviyesi kesme kuvvetini, $W$
yalıtım düzlemi dahil üstyapı kütlesini, $F_{\textrm{s}}$ üstyapı
taban kesme kuvvetini, $W_{\textrm{s}}$ ise üstyapı kütlesini temsil
etmektedir. Belirtilen yapıların bağımsız yalıtım düzlemi üzerinde
bulunması durumunda ilgili terime ``$\textrm{b}$'' alt indisi,
ortak yalıtım düzleminde bulunması durumunda ise ``$\textrm{o}$''
alt indisi getirilmiştir. Daha sonra yazılan rakamlar bina indislerini
göstermektedir. Buna göre bağımsız ve ortak yalıtım düzleminde bulunan
yapılar için kullanılan notasyonlar Şekil \ref{Notations}'de gösterilmiştir.

\begin{figure}[h]
\noindent \begin{centering}
\includegraphics{figures/SingleStructureShear}\includegraphics{figures/MultiStructureShear}
\par\end{centering}
\caption{Sismik yalıtımlı tek ve çok yapılı sistemler için kullanılan notasyonlar.}
\label{Notations}
\end{figure}

Burada üstyapı kat deplasmanları yalıtım düzlemine göre bağıl olarak
verilmiştir. Deplasman terimlerinde bulunan üst indis hareketin bağıl
olduğu konumu göstermektedir. Alt indiste bulunan ilk rakam ilgili
kat numarası, ikinci rakam ise bina indisini göstermektedir.

\subsection{Türkiye Pratiğinde Uygulanan Yöntemler}

\subsection{Uygunlaştırma Yöntemi}

Bu çalışmada seçilen deprem kayıtları Atik ve Abrahamson (2010) tarafından
önerilen spektral eşleştirme yöntemi kullanılarak ölçeklendirilmiştir.

\begin{equation}
\delta\ddot{x}_{\textrm{g}}\left(t\right)=\stackrel[j=1]{M}{\mathbf{\sum}}b_{j}f_{j}\left(t\right)
\end{equation}

\begin{equation}
\Delta R\left(T_{i}\right)=\stackrel[j=1]{M}{\mathbf{\sum}}b_{j}c_{ij}\quad\mathbf{b}=\mathbf{C}^{-1}\Delta\mathbf{R}
\end{equation}


\section{Örnek Yapı Analizleri}

\subsection{Örnek 1}

\subsubsection{Proje hakkında bilgi}

\subsubsection{Örnek blok ETABS modeli}

\subsubsection{Tüm blok özellikleri}

\subsubsection{İzolatör özellikleri}

\subsection{Depremsellik}

\subsubsection{MCE spektrumu}

\subsubsection{Deprem kayıtları ve uygunlaştırma}

\subsection{Örnek Yapıların Çubuk Modellemesi İle İlgili Çalışmalar}

\subsection{Türkiye'deki uygulama neticesinde elde edilen kuvvetler}

\subsection{NLTHA Sonuçları}

\subsection{Karşılaştırma ve Değerlendirme}

\section{Sonuç}

\subsection{Temel Sonuçlar ve Değerlendirme}

\subsection{Tavsiyeler}

\printbibliography


\section{Denklemler}

\begin{equation}
\Phi=\dfrac{12EI}{G\left(A/\alpha\right)L^{2}}=24\alpha\left(1+\upsilon\right)\left(\dfrac{r}{L}\right)^{2}
\end{equation}
Burada, $r$ atalet yarıçapını, $\upsilon$ ise Poisson oranını $G$
kayma modülünü, $A$ kesit alanını ve $\alpha$kayma alanı katsayısını
ifade etmektedir. Kesitteki üniform olmayan kayma gerilmesi dağılımlarını
dikkate almak için kullanılan $\alpha$ katsayısı aşağıda sunulmuştur.
\begin{equation}
\alpha=\dfrac{A}{I^{2}}\int_{A}\dfrac{Q\left(y\right)^{2}}{b\left(y\right)^{2}}dA
\end{equation}
(buraya parametrelerin nasıl kalibre edildiği ile ilgili bir cümle
yazılacak!!!)

\begin{equation}
\Delta\mathbf{F}_{\text{s,}i}^{\text{fictive,}j+1}=\mathbf{K}_{\text{T,}i}^{j+1}\Delta\mathbf{u}_{i}^{j+1},\quad\Delta\mathbf{u}_{i}^{j+1}=\Delta\mathbf{F}_{\text{s,}i}^{\text{res,}j+1}(\mathbf{A}+\mathbf{K}_{\text{T,}i}^{j+1})^{-1}\label{eq-assumed-Fs-delta-u-2}
\end{equation}

Newton-Raphson iterasyonunun $j+1$ adımı sonucunda elde edilen dengelenmemiş
iç kuvvetin belirlenen hata sınırları altında kalması durumunda $t$
zaman adımı için dinamik denge sağlanmaktadır.
\begin{equation}
\Delta\mathbf{F}_{\text{s,}i}^{\text{res,}j+1}=\Delta\mathbf{F}_{\text{s,}i}^{\text{fictive,}j+1}-\mathbf{F}_{\text{s,}i}^{\text{int,}j+1}\label{eq-unbalanced-f-2-1}
\end{equation}

$t$ zaman adımı için elde edilmesi gereken yer değiştirme ve iç kuvvet
değerleri Newton-Raphson iterasyonlarında bulunan sonuçların toplanması
ile bulunur. $\quad\Delta\mathbf{F}_{\text{s,}i}^{\text{iç}}=\sum\limits _{j\text{=1}}^{n}\Delta\mathbf{F}_{\text{s,}i}^{\text{iç,}j}$

\begin{equation}
-k_{2}\left[x_{2}^{\text{b}}-x_{1}^{\text{b}}\right]+k_{1}x_{1}^{\text{b}}-c_{2}\left[\dot{x}_{2}^{\text{b}}-\dot{x}_{1}^{\text{b}}\right]+c_{1}\dot{x}_{1}^{\text{b}}=-m_{1}\ddot{x}_{1}^{\text{abs}}\label{eq-eom-1}
\end{equation}

\begin{equation}
\ddot{x}_{1}^{\text{abs}}=\ddot{x}_{1}^{\text{b}}+\ddot{x}_{\text{b}}^{\text{abs}}\label{eq-acc-1}
\end{equation}

\begin{equation}
k_{2}\left[x_{2}^{\text{b}}-x_{1}^{\text{b}}\right]+c_{2}\left[\dot{x}_{2}^{\text{b}}-\dot{x}_{1}^{\text{b}}\right]=-m_{2}\ddot{x}_{2}^{\text{abs}}\label{eq-eom-2}
\end{equation}

\begin{equation}
\ddot{x}_{2}^{\text{abs}}=\ddot{x}_{2}^{\text{b}}+\ddot{x}_{\text{b}}^{\text{abs}}\label{eq-acc-2}
\end{equation}

\begin{equation}
\begin{split}\begin{bmatrix}m_{1} & 0\\
0 & m_{2}
\end{bmatrix}\begin{Bmatrix}\ddot{x}_{1}^{\text{b}}\\
\ddot{x}_{2}^{\text{b}}
\end{Bmatrix}+ & \begin{bmatrix}c_{1}+c_{2} & -c_{2}\\
-c_{2} & c_{2}
\end{bmatrix}\begin{Bmatrix}\dot{x}_{1}^{\text{b}}\\
\dot{x}_{2}^{\text{b}}
\end{Bmatrix}+\begin{bmatrix}k_{1}+k_{2} & -k_{2}\\
-k_{2} & k_{2}
\end{bmatrix}\begin{Bmatrix}{x}_{1}^{\text{b}}\\
{x}_{2}^{\text{b}}
\end{Bmatrix}=\\
 & -\begin{bmatrix}m_{1} & 0\\
0 & m_{2}
\end{bmatrix}\begin{Bmatrix}1\\
1
\end{Bmatrix}\ddot{x}_{\text{b}}^{\text{abs}}
\end{split}
\label{eq-eom-singlestructure}
\end{equation}

\begin{equation}
\ddot{x}_{\text{b}}^{\text{abs}}=\ddot{x}_{\text{b}}^{\text{g}}+\ddot{x}_{\text{g}}^{\text{abs}}\label{eq-abs-base-acc}
\end{equation}

\begin{equation}
\mathbf{M}_{\text{s}}\mathbf{\ddot{x}}_{\text{s}}^{\text{b}}+\mathbf{C}_{\text{s}}\mathbf{\dot{x}}_{\text{s}}^{\text{b}}+\mathbf{K}_{\text{s}}^{\text{b}}\mathbf{x}_{\text{s}}^{\text{b}}=-\mathbf{M}_{\text{s}}\mathbf{R}\ddot{x}_{\text{b}}^{\text{g}}-\mathbf{M}_{\text{s}}\mathbf{R}\ddot{x}_{\text{g}}^{\text{abs}}\label{eq-eom-singlestructure-closed}
\end{equation}

\begin{equation}
-F_{\text{s}}+F_{\text{iso}}=-M_{\text{b}}\ddot{x}_{\text{b}}^{\text{abs}}\label{eq-eom-base-single-1}
\end{equation}

\begin{equation}
F_{\text{s}}=-\mathbf{R}^{\text{T}}\mathbf{M}_{\text{s}}\left(\mathbf{\ddot{x}}_{\text{s}}^{\text{b}}+\mathbf{R}\ddot{x}_{\text{b}}^{\text{abs}}\right)\label{eq-fs-internal-single-1}
\end{equation}

\begin{equation}
F_{\text{s}}=-(\mathbf{R}^{\text{T}}\mathbf{M}_{\text{s}}\mathbf{\ddot{x}}_{\text{s}}^{\text{b}}+\mathbf{R}^{\text{T}}\mathbf{M}_{\text{s}}\mathbf{R}\ddot{x}_{\text{b}}^{\text{g}}+\mathbf{R}^{\text{T}}\mathbf{M}_{\text{s}}\mathbf{R}\ddot{x}_{\text{g}}^{\text{abs}})\label{eq-fs-internal-single-2}
\end{equation}

\begin{equation}
F_{\text{iso}}=F_{\text{iso}}^{\text{NL}}+K_{\text{b}}x_{\text{b}}^{\text{g}}+C_{\text{b}}\dot{x}_{\text{b}}^{\text{g}}\label{eq-fiso-single}
\end{equation}

\begin{equation}
\mathbf{R}^{\text{T}}\mathbf{M}_{\text{s}}\mathbf{\ddot{x}}_{\text{s}}^{\text{b}}+\mathbf{R}^{\text{T}}\mathbf{M}_{\text{s}}\mathbf{R}\ddot{x}_{\text{b}}^{\text{g}}+\mathbf{R}^{\text{T}}\mathbf{M}_{\text{s}}\mathbf{R}\ddot{x}_{\text{g}}^{\text{abs}}+F_{\text{iso}}^{\text{NL}}+K_{\text{b}}x_{\text{b}}^{\text{g}}+C_{\text{b}}\dot{x}_{\text{b}}^{\text{g}}=-M_{\text{b}}\ddot{x}_{\text{b}}^{\text{g}}-M_{\text{b}}\ddot{x}_{\text{g}}^{\text{abs}}\label{eq-eom-base-single-2}
\end{equation}

\begin{equation}
\begin{split}\begin{bmatrix}\mathbf{M}_{\text{s}} & \mathbf{M}_{\text{s}}\mathbf{R}\\
\mathbf{R}^{\text{T}}\mathbf{M}_{\text{s}} & \mathbf{R}^{\text{T}}\mathbf{M}_{\text{s}}\mathbf{R}+M_{\text{b}}
\end{bmatrix} & \begin{Bmatrix}\mathbf{\ddot{x}}_{\text{s}}^{\text{b}}\\
\ddot{x}_{\text{b}}^{\text{g}}
\end{Bmatrix}+\begin{bmatrix}\mathbf{C}_{\text{s}} & \mathbf{0}\\
\mathbf{0} & C_{\text{b}}
\end{bmatrix}\begin{Bmatrix}\mathbf{\dot{x}}_{\text{s}}^{\text{b}}\\
\dot{x}_{\text{b}}^{\text{g}}
\end{Bmatrix}+\begin{bmatrix}\mathbf{K}_{\text{s}} & \mathbf{0}\\
\mathbf{0} & K_{\text{b}}
\end{bmatrix}\begin{Bmatrix}\mathbf{x}_{\text{s}}^{\text{b}}\\
{x}_{\text{b}}^{\text{g}}
\end{Bmatrix}+\begin{Bmatrix}\mathbf{0}\\
1
\end{Bmatrix}F_{\text{iso}}^{\text{NL}}=\\
 & -\begin{bmatrix}\mathbf{M}_{\text{s}} & \mathbf{M}_{\text{s}}\mathbf{R}\\
\mathbf{R}^{\text{T}}\mathbf{M}_{\text{s}} & \mathbf{R}^{\text{T}}\mathbf{M}_{\text{s}}\mathbf{R}+M_{\text{b}}
\end{bmatrix}\begin{Bmatrix}\mathbf{0}\\
1
\end{Bmatrix}\ddot{x}_{\text{g}}^{\text{abs}}
\end{split}
\label{eq-eom-single}
\end{equation}

\begin{equation}
\begin{split}\begin{bmatrix}m_{1,1} & 0\\
0 & m_{1,2}
\end{bmatrix}\begin{Bmatrix}\ddot{x}_{1,1}^{\text{b}}\\
\ddot{x}_{1,2}^{\text{b}}
\end{Bmatrix}+ & \begin{bmatrix}c_{1,1}+c_{1,2} & -c_{1,2}\\
-c_{1,2} & c_{1,2}
\end{bmatrix}\begin{Bmatrix}\dot{x}_{1,1}^{\text{b}}\\
\dot{x}_{1,2}^{\text{b}}
\end{Bmatrix}+\begin{bmatrix}k_{1,1}+k_{1,2} & -k_{1,2}\\
-k_{1,2} & k_{1,2}
\end{bmatrix}\begin{Bmatrix}{x}_{1,1}^{\text{b}}\\
{x}_{1,2}^{\text{b}}
\end{Bmatrix}=\\
 & -\begin{bmatrix}m_{1,1} & 0\\
0 & m_{1,2}
\end{bmatrix}\begin{Bmatrix}1\\
1
\end{Bmatrix}(\ddot{x}_{\text{b}}^{\text{g}}+\ddot{x}_{\text{g}}^{\text{abs}})
\end{split}
\label{eq-eom-multi-s1}
\end{equation}

\begin{equation}
\mathbf{M}_{\text{s,1}}\mathbf{\ddot{x}}_{\text{s,1}}^{\text{b}}+\mathbf{C}_{\text{s,1}}\mathbf{\dot{x}}_{\text{s,1}}^{\text{b}}+\mathbf{K}_{\text{s,1}}\mathbf{x}_{\text{s,1}}^{\text{b}}=-\mathbf{M}_{\text{s,1}}\mathbf{R}_{1}(\ddot{x}_{\text{b}}^{\text{g}}+\ddot{x}_{\text{g}}^{\text{abs}})\label{eq-eom-multi-s1-closed}
\end{equation}

\begin{equation}
\begin{split} & \begin{bmatrix}m_{2,1} & 0 & 0\\
0 & m_{2,2} & 0\\
0 & 0 & m_{2,3}
\end{bmatrix}\begin{Bmatrix}\ddot{x}_{2,1}^{\text{b}}\\
\ddot{x}_{2,2}^{\text{b}}\\
\ddot{x}_{2,3}^{\text{b}}
\end{Bmatrix}+\begin{bmatrix}c_{2,1}+c_{2,2} & -c_{2,2} & 0\\
-c_{2,2} & c_{2,2}+c_{2,3} & -c_{2,3}\\
0 & -c_{2,3} & c_{2,3}
\end{bmatrix}\begin{Bmatrix}\dot{x}_{2,1}^{\text{b}}\\
\dot{x}_{2,2}^{\text{b}}\\
\dot{x}_{2,3}^{\text{b}}
\end{Bmatrix}+\\
 & \begin{bmatrix}k_{2,1}+k_{2,2} & -k_{2,2} & 0\\
-k_{2,2} & k_{2,2}+k_{2,3} & -k_{2,3}\\
0 & -k_{2,3} & k_{2,3}
\end{bmatrix}\begin{Bmatrix}{x}_{2,1}^{\text{b}}\\
{x}_{2,2}^{\text{b}}\\
{x}_{2,3}^{\text{b}}
\end{Bmatrix}=-\begin{bmatrix}m_{2,1} & 0 & 0\\
0 & m_{2,2} & 0\\
0 & 0 & m_{2,3}
\end{bmatrix}\begin{Bmatrix}1\\
1\\
1
\end{Bmatrix}(\ddot{x}_{\text{b}}^{\text{g}}+\ddot{x}_{\text{g}}^{\text{abs}})
\end{split}
\label{eq-eom-multi-s2}
\end{equation}

\begin{equation}
\mathbf{M}_{\text{s,2}}\mathbf{\ddot{x}}_{\text{s,2}}^{\text{b}}+\mathbf{C}_{\text{s,2}}\mathbf{\dot{x}}_{\text{s,2}}^{\text{b}}+\mathbf{K}_{\text{s,2}}\mathbf{x}_{\text{s,2}}^{\text{b}}=-\mathbf{M}_{\text{s,2}}\mathbf{R}_{2}(\ddot{x}_{\text{b}}^{\text{g}}+\ddot{x}_{\text{g}}^{\text{abs}})\label{eq-eom-multi-s2-closed}
\end{equation}

\begin{equation}
\begin{bmatrix}m_{3,1}\end{bmatrix}\begin{Bmatrix}\ddot{x}_{3,1}^{\text{b}}\end{Bmatrix}+\begin{bmatrix}c_{3,1}\end{bmatrix}\begin{Bmatrix}\dot{x}_{3,1}^{\text{b}}\end{Bmatrix}+\begin{bmatrix}k_{3,1}\end{bmatrix}\begin{Bmatrix}{x}_{3,1}^{\text{b}}\end{Bmatrix}=-\begin{bmatrix}m_{3,1}\end{bmatrix}\begin{Bmatrix}1\end{Bmatrix}(\ddot{x}_{\text{b}}^{\text{g}}+\ddot{x}_{\text{g}}^{\text{abs}})\label{eq-eom-multi-s3}
\end{equation}

\begin{equation}
M_{\text{s,3}}\ddot{x}_{\text{s,3}}^{\text{b}}+C_{\text{s,3}}\dot{x}_{\text{s,3}}^{\text{b}}+K_{\text{s,3}}x_{\text{s,3}}^{\text{b}}=-M_{\text{s,3}}R_{3}(\ddot{x}_{\text{b}}^{\text{g}}+\ddot{x}_{\text{g}}^{\text{abs}})\label{eq-eom-multi-s3-closed}
\end{equation}

\begin{equation}
\begin{split} & \begin{bmatrix}\mathbf{M}_{\text{s,1}} & \mathbf{0} & \mathbf{0}\\
\mathbf{0} & \mathbf{M}_{\text{s,2}} & \mathbf{0}\\
\mathbf{0} & \mathbf{0} & M_{\text{s,3}}
\end{bmatrix}\begin{Bmatrix}\mathbf{\ddot{x}}_{\text{s,1}}^{\text{b}}\\
\mathbf{\ddot{x}}_{\text{s,2}}^{\text{b}}\\
\ddot{x}_{\text{s,3}}^{\text{b}}
\end{Bmatrix}+\begin{bmatrix}\mathbf{C}_{\text{s,1}} & \mathbf{0} & \mathbf{0}\\
\mathbf{0} & \mathbf{C}_{\text{s,2}} & \mathbf{0}\\
\mathbf{0} & \mathbf{0} & C_{\text{s,3}}
\end{bmatrix}\begin{Bmatrix}\mathbf{\dot{x}}_{\text{s,1}}^{\text{b}}\\
\mathbf{\dot{x}}_{\text{s,2}}^{\text{b}}\\
\dot{x}_{\text{s,3}}^{\text{b}}
\end{Bmatrix}+\\
 & \begin{bmatrix}\mathbf{K}_{\text{s,1}} & \mathbf{0} & \mathbf{0}\\
\mathbf{0} & \mathbf{K}_{\text{s,2}} & \mathbf{0}\\
\mathbf{0} & \mathbf{0} & K_{\text{s,3}}
\end{bmatrix}\begin{Bmatrix}\mathbf{x}_{\text{s,1}}^{\text{b}}\\
\mathbf{x}_{\text{s,2}}^{\text{b}}\\
{x}_{\text{s,3}}^{\text{b}}
\end{Bmatrix}=-\begin{bmatrix}\mathbf{M}_{\text{s,1}} & \mathbf{0} & \mathbf{0}\\
\mathbf{0} & \mathbf{M}_{\text{s,2}} & \mathbf{0}\\
\mathbf{0} & \mathbf{0} & M_{\text{s,3}}
\end{bmatrix}\begin{Bmatrix}\mathbf{R}_{\text{1}}\\
\mathbf{R}_{\text{2}}\\
R_{\text{3}}
\end{Bmatrix}(\ddot{x}_{\text{b}}^{\text{g}}+\ddot{x}_{\text{g}}^{\text{abs}})
\end{split}
\label{eq-eom-multi}
\end{equation}

\begin{equation}
\mathbf{M}_{\text{s}}\mathbf{\ddot{x}}_{\text{s}}^{\text{b}}+\mathbf{C}_{\text{s}}\mathbf{\dot{x}}_{\text{s}}^{\text{b}}+\mathbf{K}_{\text{s}}\mathbf{x}_{\text{s}}^{\text{b}}=-\mathbf{M}_{\text{s}}\mathbf{R}(\ddot{x}_{\text{b}}^{\text{g}}+\ddot{x}_{\text{g}}^{\text{abs}})\label{eq-eom-multi-closed}
\end{equation}

\begin{equation}
F_{\text{s,1}}=-(\mathbf{R}_{1}^{\text{T}}\mathbf{M}_{\text{s,1}}\mathbf{\ddot{x}}_{\text{s,1}}^{\text{b}}+\mathbf{R}_{1}^{\text{T}}\mathbf{M}_{\text{s,1}}\mathbf{R}\ddot{x}_{\text{b}}^{\text{g}}+\mathbf{R}_{1}^{\text{T}}\mathbf{M}_{\text{s,1}}\mathbf{R}_{1}\ddot{x}_{\text{g}}^{\text{abs}})\label{eq-fs-multi-s1}
\end{equation}

\begin{equation}
F_{\text{s,2}}=-(\mathbf{R}_{2}^{\text{T}}\mathbf{M}_{\text{s,2}}\mathbf{\ddot{x}}_{\text{s,2}}^{\text{b}}+\mathbf{R}_{2}^{\text{T}}\mathbf{M}_{\text{s,2}}\mathbf{R}\ddot{x}_{\text{b}}^{\text{g}}+\mathbf{R}_{2}^{\text{T}}\mathbf{M}_{\text{s,2}}\mathbf{R}_{2}\ddot{x}_{\text{g}}^{\text{abs}})\label{eq-fs-multi-s2}
\end{equation}

\begin{equation}
F_{\text{s,3}}=-(\mathbf{R}_{3}^{\text{T}}\mathbf{M}_{\text{s,3}}\mathbf{\ddot{x}}_{\text{s,3}}^{\text{b}}+\mathbf{R}_{3}^{\text{T}}\mathbf{M}_{\text{s,3}}\mathbf{R}\ddot{x}_{\text{b}}^{\text{g}}+\mathbf{R}_{3}^{\text{T}}\mathbf{M}_{\text{s,3}}\mathbf{R}_{3}\ddot{x}_{\text{g}}^{\text{abs}})\label{eq-fs-multi-s3}
\end{equation}

\begin{equation}
F_{\text{s}}=-(\mathbf{R}^{\text{T}}\mathbf{M}_{\text{s}}\mathbf{\ddot{x}}_{\text{s}}^{\text{b}}+\mathbf{R}^{\text{T}}\mathbf{M}_{\text{s}}\mathbf{R}\ddot{x}_{\text{b}}^{\text{g}}+\mathbf{R}^{\text{T}}\mathbf{M}_{\text{s}}\mathbf{R}\ddot{x}_{\text{g}}^{\text{abs}})\label{eq-fs-multi}
\end{equation}

\begin{equation}
F_{\text{iso}}=F_{\text{iso}}^{\text{NL}}+K_{\text{b}}x_{\text{b}}^{\text{g}}+C_{\text{b}}\dot{x}_{\text{b}}^{\text{g}}\label{eq-fiso-multi}
\end{equation}

\begin{equation}
-F_{\text{s}}+F_{\text{iso}}=-M_{\text{b}}\ddot{x}_{\text{b}}^{\text{abs}}\label{eq-eom-base-multi-1}
\end{equation}

\begin{equation}
\mathbf{R}^{\text{T}}\mathbf{M}_{\text{s}}\ddot{x}_{\text{s}}^{\text{b}}+\mathbf{R}^{\text{T}}\mathbf{M}_{\text{s}}\mathbf{R}\ddot{x}_{\text{b}}^{\text{g}}+\mathbf{R}^{\text{T}}\mathbf{M}_{\text{s}}\mathbf{R}\ddot{x}_{\text{g}}^{\text{abs}}+F_{\text{iso}}^{\text{NL}}+K_{\text{b}}x_{\text{b}}^{\text{g}}+C_{\text{b}}\dot{x}_{\text{b}}^{\text{g}}=-M_{\text{b}}\ddot{x}_{\text{b}}^{\text{g}}-M_{\text{b}}\ddot{x}_{\text{g}}^{\text{abs}}\label{eq-eom-base-multi-2}
\end{equation}

\begin{equation}
\begin{split}\begin{bmatrix}\mathbf{M}_{\text{s}} & \mathbf{M}_{\text{s}}\mathbf{R}\\
\mathbf{R}^{\text{T}}\mathbf{M}_{\text{s}} & \mathbf{R}^{\text{T}}\mathbf{M}_{\text{s}}\mathbf{R}+M_{\text{b}}
\end{bmatrix} & \begin{Bmatrix}\mathbf{\ddot{x}}_{\text{s}}^{\text{b}}\\
\ddot{x}_{\text{b}}^{\text{g}}
\end{Bmatrix}+\begin{bmatrix}\mathbf{C}_{\text{s}} & \mathbf{0}\\
\mathbf{0} & C_{\text{b}}
\end{bmatrix}\begin{Bmatrix}\mathbf{\dot{x}}_{\text{s}}^{\text{b}}\\
\dot{x}_{\text{b}}^{\text{g}}
\end{Bmatrix}+\begin{bmatrix}\mathbf{K}_{\text{s}} & \mathbf{0}\\
\mathbf{0} & K_{\text{b}}
\end{bmatrix}\begin{Bmatrix}\mathbf{x}_{\text{s}}^{\text{b}}\\
{x}_{\text{b}}^{\text{g}}
\end{Bmatrix}+\begin{Bmatrix}\mathbf{0}\\
1
\end{Bmatrix}F_{\text{iso}}^{\text{NL}}=\\
 & -\begin{bmatrix}\mathbf{M}_{\text{s}} & \mathbf{M}_{\text{s}}\mathbf{R}\\
\mathbf{R}^{\text{T}}\mathbf{M}_{\text{s}} & \mathbf{R}^{\text{T}}\mathbf{M}_{\text{s}}\mathbf{R}+M_{\text{b}}
\end{bmatrix}\begin{Bmatrix}\mathbf{0}\\
1
\end{Bmatrix}\ddot{x}_{\text{g}}^{\text{abs}}
\end{split}
\label{eq-eom-multistructures}
\end{equation}

Hareket Denklemlerinin Doğrusal Olmayan Çözüm Yöntemi

\begin{equation}
\mathbf{M\ddot{{u}}}(t)+\mathbf{C\dot{{u}}}(t)+\mathbf{F}_{\text{s}}(t)=\mathbf{P}(t)\label{eq-eom-nl}
\end{equation}

\begin{equation}
\mathbf{M}\Delta\mathbf{\ddot{{u}}}_{i}+\mathbf{C}\Delta\mathbf{\dot{{u}}}_{i}+\Delta\mathbf{F}_{\text{s,}i}=\Delta\mathbf{P}_{i}\label{eq-eom-nl-incremental}
\end{equation}

\begin{equation}
\mathbf{\dot{{u}}}_{i+1}=\mathbf{\dot{{u}}}_{i}+[(1-\gamma)\Delta t]\mathbf{\ddot{{u}}}_{i}+(\gamma\Delta t)\mathbf{\ddot{{u}}}_{i+1}\label{eq-newmark-v}
\end{equation}

\begin{equation}
\mathbf{{u}}_{i+1}=\mathbf{{u}}_{i}+(\Delta t)\mathbf{\dot{{u}}}_{i}+[(0.5-\beta)(\Delta t)^{2}]\mathbf{\ddot{{u}}}_{i}+[\beta(\Delta t)^{2}]\mathbf{\ddot{{u}}}_{i+1}\label{eq-newmark-u}
\end{equation}

\begin{equation}
\Delta\mathbf{\dot{{u}}}_{i}=\dfrac{\gamma}{\beta\Delta t}\Delta\mathbf{{u}}_{i}-\dfrac{\gamma}{\beta}\mathbf{\dot{{u}}}_{i}+\Delta t(1-\dfrac{\gamma}{2\beta})\mathbf{\ddot{{u}}}_{i}\label{eq-newmark-v-incremental}
\end{equation}

\begin{equation}
\Delta\mathbf{\ddot{{u}}}_{i}=\dfrac{1}{\beta\Delta t^{2}}\Delta\mathbf{{u}}_{i}-\dfrac{1}{\beta\Delta t}\mathbf{\dot{{u}}}_{i}-\dfrac{1}{2\beta}\mathbf{\ddot{{u}}}_{i}\label{eq-newmark-a-incremental}
\end{equation}

\begin{equation}
\mathbf{A}\Delta\mathbf{u}_{i}+\Delta\mathbf{F}_{\text{s,}i}=\Delta\hat{\mathbf{P}}_{i}\label{eq-eom-nl-closed}
\end{equation}

\begin{equation}
\mathbf{A}=\dfrac{1}{\beta\Delta t^{2}}\mathbf{M}+\dfrac{\gamma}{\beta\Delta t}\mathbf{C}\label{eq-A}
\end{equation}

\begin{equation}
\Delta\hat{\mathbf{P}}_{i}=\Delta\mathbf{P}_{i}+(\dfrac{1}{\beta\Delta t}\mathbf{M}+\dfrac{\gamma}{\beta}\mathbf{C})\dot{u}_{i}+[\dfrac{1}{2\beta}\mathbf{M}+\Delta t(\dfrac{\gamma}{2\beta}-1)\mathbf{C}]\ddot{u}_{i}\label{eq-delta-P-hat}
\end{equation}

\begin{equation}
\Delta\mathbf{F}_{\text{s,}i}^{\text{fictive,}j}=K_{\text{T,}i}^{j}\Delta\mathbf{u}_{i}^{j}\label{eq-assumed-Fs}
\end{equation}

\begin{equation}
\Delta\mathbf{u}_{i}^{j}=\Delta\hat{\mathbf{P}}_{i}(\mathbf{A}+\mathbf{K}_{\text{T,}i}^{j})^{-1}\label{eq-delta-u}
\end{equation}

\begin{equation}
\Delta\mathbf{F}_{\text{s,}i}^{\text{res,}j}=\Delta\mathbf{F}_{\text{s,}i}^{\text{fictive,}j}-\mathbf{F}_{\text{s,}i}^{\text{int,}j}\label{eq-unbalanced-f}
\end{equation}

\begin{equation}
\mathbf{A}\Delta\mathbf{u}_{i}^{j+1}+\Delta\mathbf{F}_{\text{s,}i}^{j+1}=\Delta\mathbf{F}_{\text{s,}i}^{\text{res,}j+1}\label{eq-unbalanced-external-f}
\end{equation}

\begin{equation}
\Delta\mathbf{F}_{\text{s,}i}^{\text{fictive,}j+1}=\mathbf{K}_{\text{T,}i}^{j+1}\Delta\mathbf{u}_{i}^{j+1}\label{eq-assumed-Fs-2}
\end{equation}

\begin{equation}
\Delta\mathbf{u}_{i}^{j+1}=\Delta\mathbf{F}_{\text{s,}i}^{\text{res,}j+1}(\mathbf{A}+\mathbf{K}_{\text{T,}i}^{j+1})^{-1}\label{eq-delta-u-2}
\end{equation}

\begin{equation}
\Delta\mathbf{F}_{\text{s,}i}^{\text{res,}j+1}=\Delta\mathbf{F}_{\text{s,}i}^{\text{fictive,}j+1}-\mathbf{F}_{\text{s,}i}^{\text{int,}j+1}\label{eq-unbalanced-f-2}
\end{equation}

\begin{equation}
\Delta\mathbf{u}_{i}=\sum\limits _{j\text{=1}}^{n}\Delta\mathbf{u}_{i}^{j}\label{eq-total-u}
\end{equation}

\begin{equation}
\Delta\mathbf{F}_{\text{s,}i}^{\text{int}}=\sum\limits _{j\text{=1}}^{n}\mathbf{F}_{\text{s,}i}^{\text{int,}j}\label{eq-total-f}
\end{equation}

\begin{equation}
\mathbf{c}=a_{0}\mathbf{m}+a_{1}\mathbf{k}\label{dampingmatrix}
\end{equation}

\begin{equation}
\dfrac{1}{2}\begin{bmatrix}1/\omega_{i} & \omega_{i}\\
1/\omega_{j} & \omega_{j}
\end{bmatrix}\begin{Bmatrix}a_{0}\\
a_{1}
\end{Bmatrix}=\begin{Bmatrix}\xi_{i}\\
\xi_{j}
\end{Bmatrix}\label{dampinggeneral}
\end{equation}

\begin{equation}
a_{\text{0}}=\xi\dfrac{2\omega_{i}\omega_{j}}{\omega_{i}+\omega_{j}}\hspace{1.5cm}a_{\text{1}}=\xi\dfrac{2}{\omega_{i}+\omega_{j}}\label{dampingcoefficients}
\end{equation}

\begin{equation}
{T}_{\text{iso}}=2\pi\sqrt{\dfrac{{M}}{{k}_{\text{1}}}}\label{eq-T_yal}
\end{equation}

\begin{equation}
{T}_{\text{eff}}^{i}=3{T}_{\text{iso}}\hspace{1.5cm}\xi_{\text{eff}}^{i}=0.2\label{eq-T_eff1}
\end{equation}

\begin{equation}
{k}_{\text{eff}}^{i}=({T}_{\text{eff}}^{i})^{2}/4\pi^{2}\times M\label{eq-k_eff}
\end{equation}

\begin{equation}
\eta^{i}=\sqrt{10/(5+\xi_{\text{eff}}^{i})}\label{eq-eta-ec_1}
\end{equation}

\begin{equation}
F_{\text{s}}^{i}=M\times S_{\text{ea}}(T_{\text{eff}}^{i},\xi_{\text{eff}}^{i})\label{eq-F_s}
\end{equation}

\begin{equation}
u_{\text{max}}^{i+1}=F_{\text{s}}^{i}/{k}_{\text{eff}}^{i}\label{eq-u_max}
\end{equation}

\begin{equation}
{k}_{\text{eff}}^{i+1}=k_{2}+(u_{\text{y}}/u_{\text{max}}^{i+1})\times(k_{1}-k_{2})\label{eq-Teff}
\end{equation}

\begin{equation}
{T}_{\text{eff}}^{i+1}=2\pi\sqrt{\dfrac{{M}}{{k}_{\text{eff}}^{i+1}}}\hspace{1.5cm}\xi_{\text{eff}}^{i+1}=\dfrac{4Q(u_{\text{max}}^{i+1}-u_{\text{y}})}{2\pi k_{\text{eff}}^{i+1}(u_{\text{max}}^{i+1})^{2}}\label{eq-xieff}
\end{equation}

\begin{equation}
\eta^{i+1}=\sqrt{10/(5+\xi_{\text{eff}}^{i+1})}\label{eq-eta-ec}
\end{equation}

\begin{equation}
u_{\text{max}}^{i+1}-u_{\text{max}}^{i}<10^{-6}\label{eq-Err}
\end{equation}

\begin{equation}
f_{j}=m_{j}S_{\text{ae}}(T_{\text{eff}},\xi_{\text{eff}})\label{eq-fb-ec}
\end{equation}

\begin{equation}
V_{\text{s}}=\dfrac{k_{\text{Dmax}}D_{\text{D}}}{R_{\text{1}}}\label{eq-fb-asce7-10}
\end{equation}

\begin{equation}
V_{\text{st}}=V_{\text{b}}(\dfrac{W_{\text{s}}}{W})^{(1-a\zeta)}\label{eq-fb-asce7-16}
\end{equation}

\begin{equation}
V_{\text{M}}=\dfrac{S_{\text{ae}}^{\text{(DD-1)}}(T_{\text{M}})W\eta_{\text{M}}}{R}\label{eq-fb-tbdy2019}
\end{equation}

\begin{equation}
F_{x}=\dfrac{V_{\text{s}}w_{x}h_{x}}{\sum\limits _{i\text{=1}}^{n}w_{i}h_{i}}\label{eq-fs-asce7-10}
\end{equation}

\begin{equation}
F_{x}=\dfrac{V_{\text{st}}w_{x}h_{x}^{k_{\text{bi}}}}{\sum\limits _{i\text{=1}}^{n}w_{i}h_{i}^{k_{\text{bi}}}}\hspace{1cm}k_{bi}=14\beta_{\text{D}}T_{\text{S}}\leq4\label{eq-fs-asce41-13}
\end{equation}

\end{document}
